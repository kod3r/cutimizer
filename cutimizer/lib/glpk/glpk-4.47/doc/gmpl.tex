%* gmpl.tex *%

%***********************************************************************
%  This code is part of GLPK (GNU Linear Programming Kit).
%
%  Copyright (C) 2000, 2001, 2002, 2003, 2004, 2005, 2006, 2007, 2008,
%  2009, 2010 Andrew Makhorin, Department for Applied Informatics,
%  Moscow Aviation Institute, Moscow, Russia. All rights reserved.
%  E-mail: <mao@gnu.org>.
%
%  GLPK is free software: you can redistribute it and/or modify it
%  under the terms of the GNU General Public License as published by
%  the Free Software Foundation, either version 3 of the License, or
%  (at your option) any later version.
%
%  GLPK is distributed in the hope that it will be useful, but WITHOUT
%  ANY WARRANTY; without even the implied warranty of MERCHANTABILITY
%  or FITNESS FOR A PARTICULAR PURPOSE. See the GNU General Public
%  License for more details.
%
%  You should have received a copy of the GNU General Public License
%  along with GLPK. If not, see <http://www.gnu.org/licenses/>.
%***********************************************************************

\documentclass[10pt]{article}
\usepackage[dvipdfm,linktocpage,colorlinks,linkcolor=blue]{hyperref}

\begin{document}

\thispagestyle{empty}

\begin{center}

\vspace*{1in}

\begin{huge}
\sf\bfseries Modeling Language GNU MathProg
\end{huge}

\vspace{0.5in}

\begin{LARGE}
\sf Language Reference
\end{LARGE}

\vspace{0.5in}

\begin{LARGE}
\sf for GLPK Version 4.45
\end{LARGE}

\vspace{0.5in}
\begin{Large}
\sf (DRAFT, December 2010)
\end{Large}

\end{center}

\newpage

\vspace*{1in}

\vfill

\noindent
The GLPK package is part of the GNU Project released under the aegis of
GNU.

\medskip\noindent
Copyright \copyright{} 2000, 2001, 2002, 2003, 2004, 2005, 2006, 2007,
2008, 2009, 2010 Andrew Makhorin, Department for Applied Informatics,
Moscow Aviation Institute, Moscow, Russia. All rights reserved.

\medskip\noindent
Free Software Foundation, Inc., 51 Franklin St, Fifth Floor, Boston,
MA 02110-1301, USA.

\medskip\noindent
Permission is granted to make and distribute verbatim copies of this
manual provided the copyright notice and this permission notice are
preserved on all copies.

\medskip\noindent
Permission is granted to copy and distribute modified versions of this
manual under the conditions for verbatim copying, provided also that
the entire resulting derived work is distributed under the terms of
a permission notice identical to this one.

\medskip\noindent
Permission is granted to copy and distribute translations of this
manual into another language, under the above conditions for modified
versions.

\newpage

\tableofcontents

%%%%%%%%%%%%%%%%%%%%%%%%%%%%%%%%%%%%%%%%%%%%%%%%%%%%%%%%%%%%%%%%%%%%%%%%

\newpage

\section{Introduction}

{\it GNU MathProg} is a modeling language intended for describing
linear mathematical programming models.\footnote{The GNU MathProg
language is a subset of the AMPL language. Its GLPK implementation is
mainly based on the paper: {\it Robert Fourer}, {\it David M. Gay}, and
{\it Brian W. Kernighan}, ``A Modeling Language for Mathematical
Programming.'' {\it Management Science} 36 (1990)\linebreak pp. 519-54.}

Model descriptions written in the GNU MathProg language consist of
a set of statements and data blocks constructed by the user from the
language elements described in this document.

In a process called {\it translation}, a program called the {\it model
translator} analyzes the model description and translates it into
internal data structures, which may be then used either for generating
mathematical programming problem instance or directly by a program
called the {\it solver} to obtain numeric solution of the problem.

\subsection{Linear programming problem}
\label{problem}

In MathProg the linear programming (LP) problem is stated as follows:

\medskip

\noindent\hspace{.7in}minimize (or maximize)
$$z=c_1x_1+c_2x_2+\dots+c_nx_n+c_0\eqno(1.1)$$
\noindent\hspace{.7in}subject to linear constraints
$$
\begin{array}{l@{\ }c@{\ }r@{\ }c@{\ }r@{\ }c@{\ }r@{\ }c@{\ }l}
L_1&\leq&a_{11}x_1&+&a_{12}x_2&+\dots+&a_{1n}x_n&\leq&U_1\\
L_2&\leq&a_{21}x_1&+&a_{22}x_2&+\dots+&a_{2n}x_n&\leq&U_2\\
\multicolumn{9}{c}{.\ \ .\ \ .\ \ .\ \ .\ \ .\ \ .\ \ .\ \ .}\\
L_m&\leq&a_{m1}x_1&+&a_{m2}x_2&+\dots+&a_{mn}x_n&\leq&U_m\\
\end{array}\eqno(1.2)
$$
\noindent\hspace{.7in}and bounds of variables
$$
\begin{array}{l@{\ }c@{\ }c@{\ }c@{\ }l}
l_1&\leq&x_1&\leq&u_1\\
l_2&\leq&x_2&\leq&u_2\\
\multicolumn{5}{c}{.\ \ .\ \ .\ \ .\ \ .}\\
l_n&\leq&x_n&\leq&u_n\\
\end{array}\eqno(1.3)
$$
where $x_1$, $x_2$, \dots, $x_n$ are variables; $z$ is the objective
function; $c_1$, $c_2$, \dots, $c_n$ are objective coefficients; $c_0$
is the constant term (``shift'') of the objective function; $a_{11}$,
$a_{12}$, \dots, $a_{mn}$ are constraint coefficients; $L_1$, $L_2$,
\dots, $L_m$ are lower constraint bounds; $U_1$, $U_2$, \dots, $U_m$
are upper constraint bounds; $l_1$, $l_2$, \dots, $l_n$ are lower
bounds of variables; $u_1$, $u_2$, \dots, $u_n$ are upper bounds of
variables.

Bounds of variables and constraint bounds can be finite as well as
infinite. Besides, lower bounds can be equal to corresponding upper
bounds. Thus, the following types of variables and constraints are
allowed:

\newpage

\begin{tabular}{@{}r@{\ }c@{\ }c@{\ }c@{\ }l@{\hspace*{38pt}}l}
$-\infty$&$<$&$x$&$<$&$+\infty$&Free (unbounded) variable\\
$l$&$\leq$&$x$&$<$&$+\infty$&Variable with lower bound\\
$-\infty$&$<$&$x$&$\leq$&$u$&Variable with upper bound\\
$l$&$\leq$&$x$&$\leq$&$u$&Double-bounded variable\\
$l$&$=$&$x$&=&$u$&Fixed variable\\
\end{tabular}

\bigskip

\begin{tabular}{@{}r@{\ }c@{\ }c@{\ }c@{\ }ll}
$-\infty$&$<$&$\sum a_jx_j$&$<$&$+\infty$&Free (unbounded) linear
form\\
$L$&$\leq$&$\sum a_jx_j$&$<$&$+\infty$&Inequality constraint ``greater
than or equal to''\\
$-\infty$&$<$&$\sum a_jx_j$&$\leq$&$U$&Inequality constraint ``less
than or equal to''\\
$L$&$\leq$&$\sum a_jx_j$&$\leq$&$U$&Double-bounded inequality
constraint\\
$L$&$=$&$\sum a_jx_j$&=&$U$&Equality constraint\\
\end{tabular}

\bigskip

In addition to pure LP problems MathProg also allows mixed integer
linear programming (MIP) problems, where some or all variables are
restricted to be integer or binary.

\subsection{Model objects}

In MathProg the model is described in terms of sets, parameters,
variables, constraints, and objectives, which are called {\it model
objects}.

The user introduces particular model objects using the language
statements. Each model object is provided with a symbolic name that
uniquely identifies the object and is intended for referencing purposes.

Model objects, including sets, can be multidimensional arrays built
over indexing sets. Formally, $n$-dimensional array $A$ is the mapping:
$$A:\Delta\rightarrow\Xi,\eqno(1.4)$$
where $\Delta\subseteq S_1\times\dots\times S_n$ is a subset of the
Cartesian product of indexing sets,\linebreak $\Xi$ is a set of array members.
In MathProg the set $\Delta$ is called the {\it subscript domain}. Its
members are $n$-tuples $(i_1,\dots,i_n)$, where $i_1\in S_1$, \dots,
$i_n\in S_n$.

If $n=0$, the Cartesian product above has exactly one member (namely,
\linebreak 0-tuple), so it is convenient to think scalar objects as
0-dimensional arrays having one member.

The type of array members is determined by the type of corresponding
model object as follows:

\medskip

\noindent\hfil
\begin{tabular}{@{}ll@{}}
Model object&Array member\\
\hline
Set&Elemental plain set\\
Parameter&Number or symbol\\
Variable&Elemental variable\\
Constraint&Elemental constraint\\
Objective&Elemental objective\\
\end{tabular}

\medskip

In order to refer to a particular object member the object should be
provided with {\it subscripts}. For example, if $a$ is a 2-dimensional
parameter defined over $I\times J$, a reference to its particular
member can be written as $a[i,j]$, where $i\in I$ and $j\in J$. It is
understood that scalar objects being 0-dimensional need no subscripts.

\subsection{Structure of model description}

It is sometimes desirable to write a model which, at various points,
may require different data for each problem instance to be solved using
that model. For this reason in MathProg the model description consists
of two parts: the {\it model section} and the {\it data section}.

The model section is a main part of the model description that contains
declarations of model objects and is common for all problems based on
the corresponding model.

The data section is an optional part of the model description that
contains data specific for a particular problem instance.

Depending on what is more convenient the model and data sections can be
placed either in one file or in two separate files. The latter feature
allows having arbitrary number of different data sections to be used
with the same model section.

%%%%%%%%%%%%%%%%%%%%%%%%%%%%%%%%%%%%%%%%%%%%%%%%%%%%%%%%%%%%%%%%%%%%%%%%

\newpage

\section{Coding model description}
\label{coding}

The model description is coded in plain text format using ASCII
character set. Characters valid in the model description are the
following:

\begin{itemize}
\item alphabetic characters:\\
\verb|A B C D E F G H I J K L M N O P Q R S T U V W X Y Z|\\
\verb|a b c d e f g h i j k l m n o p q r s t u v w x y z _|
\item numeric characters:\\
\verb|0 1 2 3 4 5 6 7 8 9|
\item special characters:\\
\verb?! " # & ' ( ) * + , - . / : ; < = > [ ] ^ { | }?
\item white-space characters:\\
\verb|SP HT CR NL VT FF|
\end{itemize}

Within string literals and comments any ASCII characters (except
control characters) are valid.

White-space characters are non-significant. They can be used freely
between lexical units to improve readability of the model description.
They are also used to separate lexical units from each other if there
is no other way to do that.

Syntactically model description is a sequence of lexical units in the
following categories:

\begin{itemize}
\item symbolic names;
\item numeric literals;
\item string literals;
\item keywords;
\item delimiters;
\item comments.
\end{itemize}

The lexical units of the language are discussed below.

\subsection{Symbolic names}

A {\it symbolic name} consists of alphabetic and numeric characters,
the first of which must be alphabetic. All symbolic names are distinct
(case sensitive).

\medskip

\noindent{\bf Examples}

\medskip

\noindent\verb|alpha123|

\noindent\verb|This_is_a_name|

\noindent\verb|_P123_abc_321|

\newpage

Symbolic names are used to identify model objects (sets, parameters,
variables, constraints, objectives) and dummy indices.

All symbolic names (except names of dummy indices) must be unique, i.e.
the model description must have no objects with identical names.
Symbolic names of dummy indices must be unique within the scope, where
they are valid.

\subsection{Numeric literals}

A {\it numeric literal} has the form {\it xx}{\tt E}{\it syy}, where
{\it xx} is a number with optional decimal point, {\it s} is the sign
{\tt+} or {\tt-}, {\it yy} is a decimal exponent. The letter {\tt E} is
case insensitive and can be coded as {\tt e}.

\medskip

\noindent{\bf Examples}

\medskip

\noindent\verb|123|

\noindent\verb|3.14159|

\noindent\verb|56.E+5|

\noindent\verb|.78|

\noindent\verb|123.456e-7|

\medskip

Numeric literals are used to represent numeric quantities. They have
obvious fixed meaning.

\subsection{String literals}

A {\it string literal} is a sequence of arbitrary characters enclosed
either in single quotes or in double quotes. Both these forms are
equivalent.

If the single quote is part of a string literal enclosed in single
quotes, it must be coded twice. Analogously, if the double quote is
part of a string literal enclosed in double quotes, it must be coded
twice.

\medskip

\noindent{\bf Examples}

\medskip

\noindent\verb|'This is a string'|

\noindent\verb|"This is another string"|

\noindent\verb|'1 + 2 = 3'|

\noindent\verb|'That''s all'|

\noindent\verb|"She said: ""No"""|

\medskip

String literals are used to represent symbolic quantities.

\subsection{Keywords}

A {\it keyword} is a sequence of alphabetic characters and possibly
some special characters.

All keywords fall into two categories: {\it reserved keywords}, which
cannot be used as symbolic names, and {\it non-reserved keywords},
which being recognized by context can be used as symbolic names.

\newpage

The reserved keywords are the following:

\medskip

\noindent\hfil
\begin{tabular}{@{}p{.7in}p{.7in}p{.7in}p{.7in}@{}}
{\tt and}&{\tt else}&{\tt mod}&{\tt union}\\
{\tt by}&{\tt if}&{\tt not}&{\tt within}\\
{\tt cross}&{\tt in}&{\tt or}\\
{\tt diff}&{\tt inter}&{\tt symdiff}\\
{\tt div}&{\tt less}&{\tt then}\\
\end{tabular}

\medskip

Non-reserved keywords are described in following sections.

All the keywords have fixed meaning, which will be explained on
discussion of corresponding syntactic constructions, where the keywords
are used.

\subsection{Delimiters}

A {\it delimiter} is either a single special character or a sequence of
two special characters as follows:

\medskip

\noindent\hfil
\begin{tabular}{@{}p{.3in}p{.3in}p{.3in}p{.3in}p{.3in}p{.3in}@{}}
{\tt+}&{\tt\textasciicircum}&{\tt==}&{\tt!}&{\tt:}&{\tt)}\\
{\tt-}&{\tt\&}&{\tt>=}&{\tt\&\&}&{\tt;}&{\tt[}\\
{\tt*}&{\tt<}&{\tt>}&{\tt||}&{\tt:=}&{\tt|}\\
{\tt/}&{\tt<=}&{\tt<>}&{\tt.}&{\tt..}&{\tt\{}\\
{\tt**}&{\tt=}&{\tt!=}&{\tt,}&{\tt(}&{\tt\}}\\
\end{tabular}

\medskip

If the delimiter consists of two characters, there must be no spaces
between the characters.

All the delimiters have fixed meaning, which will be explained on
discussion corresponding syntactic constructions, where the delimiters
are used.

\subsection{Comments}

For documenting purposes the model description can be provided with
{\it comments}, which may have two different forms. The first form is
a {\it single-line comment}, which begins with the character {\tt\#}
and extends until end of line. The second form is a {\it comment
sequence}, which is a sequence of any characters enclosed within
{\tt/*} and {\tt*/}.

\medskip

\noindent{\bf Examples}

\medskip

\noindent\verb|param n := 10; # This is a comment|

\noindent\verb|/* This is another comment */|

\medskip

Comments are ignored by the model translator and can appear anywhere in
the model description, where white-space characters are allowed.

%%%%%%%%%%%%%%%%%%%%%%%%%%%%%%%%%%%%%%%%%%%%%%%%%%%%%%%%%%%%%%%%%%%%%%%%

\newpage

\section{Expressions}

An {\it expression} is a rule for computing a value. In model
description expressions are used as constituents of certain statements.

In general case expressions consist of operands and operators.

Depending on the type of the resultant value all expressions fall into
the following categories:

\begin{itemize}
\item numeric expressions;
\item symbolic expressions;
\item indexing expressions;
\item set expressions;
\item logical expressions;
\item linear expressions.
\end{itemize}

\subsection{Numeric expressions}

A {\it numeric expression} is a rule for computing a single numeric
value represented as a floating-point number.

The primary numeric expression may be a numeric literal, dummy index,
unsubscripted parameter, subscripted parameter, built-in function
reference, iterated numeric expression, conditional numeric expression,
or another numeric expression enclosed in parentheses.

\medskip

\noindent{\bf Examples}

\medskip

\noindent
\begin{tabular}{@{}ll@{}}
\verb|1.23|&(numeric literal)\\
\verb|j|&(dummy index)\\
\verb|time|&(unsubscripted parameter)\\
\verb|a['May 2003',j+1]|&(subscripted parameter)\\
\verb|abs(b[i,j])|&(function reference)\\
\verb|sum{i in S diff T} alpha[i] * b[i,j]|&(iterated expression)\\
\verb|if i in I then 2 * p else q[i+1]|&(conditional expression)\\
\verb|(b[i,j] + .5 * c)|&(parenthesized expression)\\
\end{tabular}

\medskip

More general numeric expressions containing two or more primary numeric
expressions may be constructed by using certain arithmetic operators.

\medskip

\noindent{\bf Examples}

\medskip

\noindent\verb|j+1|

\noindent\verb|2 * a[i-1,j+1] - b[i,j]|

\noindent\verb|sum{j in J} a[i,j] * x[j] + sum{k in K} b[i,k] * x[k]|

\noindent\verb|(if i in I then 2 * p else q[i+1]) / (a[i,j] + 1.5)|

\subsubsection{Numeric literals}

If the primary numeric expression is a numeric literal, the resultant
value is obvious.

\subsubsection{Dummy indices}

If the primary numeric expression is a dummy index, the resultant value
is current value assigned to that dummy index.

\subsubsection{Unsubscripted parameters}

If the primary numeric expression is an unsubscripted parameter (which
must be 0-dimensional), the resultant value is the value of that
parameter.

\subsubsection{Subscripted parameters}

The primary numeric expression, which refers to a subscripted parameter,
has the following syntactic form:

\medskip

\noindent\hfil
{\it name}{\tt[}$i_1${\tt,} $i_2${\tt,} \dots{\tt,} $i_n${\tt]}

\medskip

\noindent where {\it name} is the symbolic name of the parameter,
$i_1$, $i_2$, \dots, $i_n$ are subscripts.

Each subscript must be a numeric or symbolic expression. The number of
subscripts in the subscript list must be the same as the dimension of
the parameter with which the subscript list is associated.

Actual values of subscript expressions are used to identify
a particular member of the parameter that determines the resultant
value of the primary expression.

\subsubsection{Function references}

In MathProg there exist the following built-in functions which may be
used in numeric expressions:

\medskip

\begin{tabular}{@{}p{96pt}p{222pt}@{}}
{\tt abs(}$x${\tt)}&$|x|$, absolute value of $x$\\
{\tt atan(}$x${\tt)}&$\arctan x$, principal value of the arc tangent of
$x$ (in radians)\\
{\tt atan(}$y${\tt,} $x${\tt)}&$\arctan y/x$, principal value of the
arc tangent of $y/x$ (in radians). In this case the signs of both
arguments $y$ and $x$ are used to determine the quadrant of the
resultant value\\
{\tt card(}$X${\tt)}&$|X|$, cardinality (the number of elements) of
set $X$\\
{\tt ceil(}$x${\tt)}&$\lceil x\rceil$, smallest integer not less than
$x$ (``ceiling of $x$'')\\
{\tt cos(}$x${\tt)}&$\cos x$, cosine of $x$ (in radians)\\
{\tt exp(}$x${\tt)}&$e^x$, base-$e$ exponential of $x$\\
{\tt floor(}$x${\tt)}&$\lfloor x\rfloor$, largest integer not greater
than $x$ (``floor of $x$'')\\
\end{tabular}

\begin{tabular}{@{}p{96pt}p{222pt}@{}}
{\tt gmtime()}&the number of seconds elapsed since 00:00:00~Jan~1, 1970,
Coordinated Universal Time (for details see Subsection \ref{gmtime},
page \pageref{gmtime})\\
{\tt length(}$s${\tt)}&$|s|$, length of character string $s$\\
{\tt log(}$x${\tt)}&$\log x$, natural logarithm of $x$\\
{\tt log10(}$x${\tt)}&$\log_{10}x$, common (decimal) logarithm of $x$\\
{\tt max(}$x_1${\tt,} $x_2${\tt,} \dots{\tt,} $x_n${\tt)}&the largest
of values $x_1$, $x_2$, \dots, $x_n$\\
{\tt min(}$x_1${\tt,} $x_2${\tt,} \dots{\tt,} $x_n${\tt)}&the smallest
of values $x_1$, $x_2$, \dots, $x_n$\\
{\tt round(}$x${\tt)}&rounding $x$ to nearest integer\\
{\tt round(}$x${\tt,} $n${\tt)}&rounding $x$ to $n$ fractional decimal
digits\\
{\tt sin(}$x${\tt)}&$\sin x$, sine of $x$ (in radians)\\
{\tt sqrt(}$x${\tt)}&$\sqrt{x}$, non-negative square root of $x$\\
{\tt str2time(}$s${\tt,} $f${\tt)}&converting character string $s$ to
calendar time (for details see Subsection \ref{str2time}, page
\pageref{str2time})\\
{\tt trunc(}$x${\tt)}&truncating $x$ to nearest integer\\
{\tt trunc(}$x${\tt,} $n${\tt)}&truncating $x$ to $n$ fractional
decimal digits\\
{\tt Irand224()}&generating pseudo-random integer uniformly distributed
in $[0,2^{24})$\\
{\tt Uniform01()}&generating pseudo-random number uniformly distributed
in $[0,1)$\\
{\tt Uniform(}$a${\tt,} $b${\tt)}&generating pseudo-random number
uniformly distributed in $[a,b)$\\
{\tt Normal01()}&generating Gaussian pseudo-random variate with
$\mu=0$ and $\sigma=1$\\
{\tt Normal(}$\mu${\tt,} $\sigma${\tt)}&generating Gaussian
pseudo-random variate with given $\mu$ and $\sigma$\\
\end{tabular}

\medskip

Arguments of all built-in functions, except {\tt card}, {\tt length},
and {\tt str2time}, must be numeric expressions. The argument of
{\tt card} must be a set expression. The argument of {\tt length} and
both arguments of {\tt str2time} must be symbolic expressions.

The resultant value of the numeric expression, which is a function
reference, is the result of applying the function to its argument(s).

Note that each pseudo-random generator function has a latent argument
(i.e. some internal state), which is changed whenever the function has
been applied. Thus, if the function is applied repeatedly even to
identical arguments, due to the side effect different resultant values
are always produced.

\subsubsection{Iterated expressions}
\label{itexpr}

An {\it iterated numeric expression} is a primary numeric expression,
which has the following syntactic form:

\medskip

\noindent\hfil
{\it iterated-operator indexing-expression integrand}

\medskip

\noindent where {\it iterated-operator} is the symbolic name of the
iterated operator to be performed (see below), {\it indexing-expression}
is an indexing expression which introduces dummy indices and controls
iterating, {\it integrand} is a numeric expression that participates in
the operation.

In MathProg there exist four iterated operators, which may be used in
numeric expressions:

\medskip

\noindent\hfil
\begin{tabular}{@{}lll@{}}
{\tt sum}&summation&$\displaystyle\sum_{(i_1,\dots,i_n)\in\Delta}
f(i_1,\dots,i_n)$\\
{\tt prod}&production&$\displaystyle\prod_{(i_1,\dots,i_n)\in\Delta}
f(i_1,\dots,i_n)$\\
{\tt min}&minimum&$\displaystyle\min_{(i_1,\dots,i_n)\in\Delta}
f(i_1,\dots,i_n)$\\
{\tt max}&maximum&$\displaystyle\max_{(i_1,\dots,i_n)\in\Delta}
f(i_1,\dots,i_n)$\\
\end{tabular}

\medskip

\noindent where $i_1$, \dots, $i_n$ are dummy indices introduced in
the indexing expression, $\Delta$ is the domain, a set of $n$-tuples
specified by the indexing expression which defines particular values
assigned to the dummy indices on performing the iterated operation,
$f(i_1,\dots,i_n)$ is the integrand, a numeric expression whose
resultant value depends on the dummy indices.

The resultant value of an iterated numeric expression is the result of
applying of the iterated operator to its integrand over all $n$-tuples
contained in the domain.

\subsubsection{Conditional expressions}
\label{ifthen}

A {\it conditional numeric expression} is a primary numeric expression,
which has one of the following two syntactic forms:

\medskip

\noindent\hfil
{\tt if} $b$ {\tt then} $x$ {\tt else} $y$

\medskip

\noindent\hspace{126.5pt}
{\tt if} $b$ {\tt then} $x$

\medskip

\noindent where $b$ is an logical expression, $x$ and $y$ are numeric
expressions.

The resultant value of the conditional expression depends on the value
of the logical expression that follows the keyword {\tt if}. If it
takes on the value {\it true}, the value of the conditional expression
is the value of the expression that follows the keyword {\tt then}.
Otherwise, if the logical expression takes on the value {\it false},
the value of the conditional expression is the value of the expression
that follows the keyword {\it else}. If the second, reduced form of the
conditional expression is used and the logical expression takes on the
value {\it false}, the resultant value of the conditional expression is
zero.

\subsubsection{Parenthesized expressions}

Any numeric expression may be enclosed in parentheses that
syntactically makes it a primary numeric expression.

Parentheses may be used in numeric expressions, as in algebra, to
specify the desired order in which operations are to be performed.
Where parentheses are used, the expression within the parentheses is
evaluated before the resultant value is used.

The resultant value of the parenthesized expression is the same as the
value of the expression enclosed within parentheses.

\subsubsection{Arithmetic operators}

In MathProg there exist the following arithmetic operators, which may
be used in numeric expressions:

\medskip

\begin{tabular}{@{}p{96pt}p{222pt}@{}}
{\tt +} $x$&unary plus\\
{\tt -} $x$&unary minus\\
$x$ {\tt +} $y$&addition\\
$x$ {\tt -} $y$&subtraction\\
$x$ {\tt less} $y$&positive difference (if $x<y$ then 0 else $x-y$)\\
$x$ {\tt *} $y$&multiplication\\
$x$ {\tt /} $y$&division\\
$x$ {\tt div} $y$&quotient of exact division\\
$x$ {\tt mod} $y$&remainder of exact division\\
$x$ {\tt **} $y$, $x$ {\tt\textasciicircum} $y$&exponentiation (raising
to power)\\
\end{tabular}

\medskip

\noindent where $x$ and $y$ are numeric expressions.

If the expression includes more than one arithmetic operator, all
operators are performed from left to right according to the hierarchy
of operations (see below) with the only exception that the
exponentiaion operators are performed from right to left.

The resultant value of the expression, which contains arithmetic
operators, is the result of applying the operators to their operands.

\subsubsection{Hierarchy of operations}
\label{hierarchy}

The following list shows the hierarchy of operations in numeric
expressions:

\medskip

\noindent\hfil
\begin{tabular}{@{}ll@{}}
Operation&Hierarchy\\
\hline
Evaluation of functions ({\tt abs}, {\tt ceil}, etc.)&1st\\
Exponentiation ({\tt**}, {\tt\textasciicircum})&2nd\\
Unary plus and minus ({\tt+}, {\tt-})&3rd\\
Multiplication and division ({\tt*}, {\tt/}, {\tt div}, {\tt mod})&4th\\
Iterated operations ({\tt sum}, {\tt prod}, {\tt min}, {\tt max})&5th\\
Addition and subtraction ({\tt+}, {\tt-}, {\tt less})&6th\\
Conditional evaluation ({\tt if} \dots {\tt then} \dots {\tt else})&
7th\\
\end{tabular}

\medskip

This hierarchy is used to determine which of two consecutive operations
is performed first. If the first operator is higher than or equal to
the second, the first operation is performed. If it is not, the second
operator is compared to the third, etc. When the end of the expression
is reached, all of the remaining operations are performed in the
reverse order.

\newpage

\subsection{Symbolic expressions}

A {\it symbolic expression} is a rule for computing a single symbolic
value represented as a character string.

The primary symbolic expression may be a string literal, dummy index,
unsubscripted parameter, subscripted parameter, built-in function
reference, conditional symbolic expression, or another symbolic
expression enclosed in parentheses.

It is also allowed to use a numeric expression as the primary symbolic
expression, in which case the resultant value of the numeric expression
is automatically converted to the symbolic type.

\medskip

\noindent{\bf Examples}

\medskip

\noindent
\begin{tabular}{@{}ll@{}}
\verb|'May 2003'|&(string literal)\\
\verb|j|&(dummy index)\\
\verb|p|&(unsubscripted parameter)\\
\verb|s['abc',j+1]|&(subscripted parameter)\\
\verb|substr(name[i],k+1,3)|&(function reference)\\
\verb|if i in I then s[i,j] else t[i+1]|&(conditional expression)\\
\verb|((10 * b[i,j]) & '.bis')|&(parenthesized expression)\\
\end{tabular}

\medskip

More general symbolic expressions containing two or more primary
symbolic expressions may be constructed by using the concatenation
operator.

\medskip

\noindent{\bf Examples}

\medskip

\noindent\verb|'abc[' & i & ',' & j & ']'|

\noindent\verb|"from " & city[i] & " to " & city[j]|

\medskip

The principles of evaluation of symbolic expressions are completely
analogous to the ones given for numeric expressions (see above).

\subsubsection{Function references}

In MathProg there exist the following built-in functions which may be
used in symbolic expressions:

\medskip

\begin{tabular}{@{}p{96pt}p{222pt}@{}}
{\tt substr(}$s${\tt,} $x${\tt)}&substring of $s$ starting from
position $x$\\
{\tt substr(}$s${\tt,} $x${\tt,} $y${\tt)}&substring of $s$ starting
from position $x$ and having length $y$\\
{\tt time2str(}$t${\tt,} $f${\tt)}&converting calendar time to
character string (for details see Subsection \ref{time2str}, page
\pageref{time2str})\\
\end{tabular}

\medskip

The first argument of {\tt substr} must be a symbolic expression while
its second and optional third arguments must be numeric expressions.

The first argument of {\tt time2str} must be a numeric expression, and
its second argument must be a symbolic expression.

The resultant value of the symbolic expression, which is a function
reference, is the result of applying the function to its arguments.

\subsubsection{Symbolic operators}

Currently in MathProg there exists the only symbolic operator:

\medskip

\noindent\hfil
{\tt s \& t}

\medskip

\noindent where $s$ and $t$ are symbolic expressions. This operator
means concatenation of its two symbolic operands, which are character
strings.

\subsubsection{Hierarchy of operations}

The following list shows the hierarchy of operations in symbolic
expressions:

\medskip

\noindent\hfil
\begin{tabular}{@{}ll@{}}
Operation&Hierarchy\\
\hline
Evaluation of numeric operations&1st-7th\\
Concatenation ({\tt\&})&8th\\
Conditional evaluation ({\tt if} \dots {\tt then} \dots {\tt else})&
7th\\
\end{tabular}

\medskip

This hierarchy has the same meaning as was explained above for numeric
expressions (see Subsection \ref{hierarchy}, page \pageref{hierarchy}).

\subsection{Indexing expressions and dummy indices}
\label{indexing}

An {\it indexing expression} is an auxiliary construction, which
specifies a plain set of $n$-tuples and introduces dummy indices. It
has two syntactic forms:

\medskip

\noindent\hspace{73.5pt}
{\tt\{} {\it entry}$_1${\tt,} {\it entry}$_2${\tt,} \dots{\tt,}
{\it entry}$_m$ {\tt\}}

\medskip

\noindent\hfil
{\tt\{} {\it entry}$_1${\tt,} {\it entry}$_2${\tt,} \dots{\tt,}
{\it entry}$_m$ {\tt:} {\it predicate} {\tt\}}

\medskip

\noindent where {\it entry}{$_1$}, {\it entry}{$_2$}, \dots,
{\it entry}{$_m$} are indexing entries, {\it predicate} is a logical
expression that specifies an optional predicate (logical condition).

Each {\it indexing entry} in the indexing expression has one of the
following three forms:

\medskip

\noindent\hspace{123pt}
$i$ {\tt in} $S$

\medskip

\noindent\hfil
{\tt(}$i_1${\tt,} $i_2${\tt,} \dots{\tt,}$i_n${\tt)} {\tt in} $S$

\medskip

\noindent\hspace{123pt}
$S$

\medskip

\noindent where $i_1$, $i_2$, \dots, $i_n$ are indices, $S$ is a set
expression (discussed in the next section) that specifies the basic set.

The number of indices in the indexing entry must be the same as the
dimension of the basic set $S$, i.e. if $S$ consists of 1-tuples, the
first form must be used, and if $S$ consists of $n$-tuples, where
$n>1$, the second form must be used.

If the first form of the indexing entry is used, the index $i$ can be
a dummy index only (see below). If the second form is used, the indices
$i_1$, $i_2$, \dots, $i_n$ can be either dummy indices or some numeric
or symbolic expressions, where at least one index must be a dummy index.
The third, reduced form of the indexing entry has the same effect as if
there were $i$ (if $S$ is 1-dimensional) or $i_1$, $i_2$, \dots, $i_n$
(if $S$ is $n$-dimensional) all specified as dummy indices.

A {\it dummy index} is an auxiliary model object, which acts like an
individual variable. Values assigned to dummy indices are components of
$n$-tuples from basic sets, i.e. some numeric and symbolic quantities.

For referencing purposes dummy indices can be provided with symbolic
names. However, unlike other model objects (sets, parameters, etc.)
dummy indices need not be explicitly declared. Each {\it undeclared}
symbolic name being used in the indexing position of an indexing entry
is recognized as the symbolic name of corresponding dummy index.

Symbolic names of dummy indices are valid only within the scope of the
indexing expression, where the dummy indices were introduced. Beyond
the scope the dummy indices are completely inaccessible, so the same
symbolic names may be used for other purposes, in particular, to
represent dummy indices in other indexing expressions.

The scope of indexing expression, where implicit declarations of dummy
indices are valid, depends on the context, in which the indexing
expression is used:

\begin{enumerate}
\item If the indexing expression is used in iterated operator, its
scope extends until the end of the integrand.
\item If the indexing expression is used as a primary set expression,
its scope extends until the end of that indexing expression.
\item If the indexing expression is used to define the subscript domain
in declarations of some model objects, its scope extends until the end
of the corresponding statement.
\end{enumerate}

The indexing mechanism implemented by means of indexing expressions is
best explained by some examples discussed below.

Let there be given three sets:

\medskip

\noindent\hspace{33.5pt}
$A=\{4,7,9\}$,

\medskip

\noindent\hfil
$B=\{(1,Jan),(1,Feb),(2,Mar),(2,Apr),(3,May),(3,Jun)\}$,

\medskip

\noindent\hspace{33.5pt}
$C=\{a,b,c\}$,

\medskip

\noindent where $A$ and $C$ consist of 1-tuples (singlets), $B$
consists of 2-tuples (doublets). Consider the following indexing
expression:

\medskip

\noindent\hfil
{\tt\{i in A, (j,k) in B, l in C\}}

\medskip

\noindent where {\tt i}, {\tt j}, {\tt k}, and {\tt l} are dummy
indices.

Although MathProg is not a procedural language, for any indexing
expression an equivalent algorithmic description can be given. In
particular, the algorithmic description of the indexing expression
above could look like follows:

\medskip

\noindent\hfil
\begin{tabular}{@{}l@{}}
{\bf for all} $i\in A$ {\bf do}\\
\hspace{12pt}{\bf for all} $(j,k)\in B$ {\bf do}\\
\hspace{24pt}{\bf for all} $l\in C$ {\bf do}\\
\hspace{36pt}{\it action};\\
\end{tabular}

\newpage

\noindent where the dummy indices $i$, $j$, $k$, $l$ are consecutively
assigned corresponding components of $n$-tuples from the basic sets $A$,
$B$, $C$, and {\it action} is some action that depends on the context,
where the indexing expression is used. For example, if the action were
printing current values of dummy indices, the printout would look like
follows:

\medskip

\noindent\hfil
\begin{tabular}{@{}llll@{}}
$i=4$&$j=1$&$k=Jan$&$l=a$\\
$i=4$&$j=1$&$k=Jan$&$l=b$\\
$i=4$&$j=1$&$k=Jan$&$l=c$\\
$i=4$&$j=1$&$k=Feb$&$l=a$\\
$i=4$&$j=1$&$k=Feb$&$l=b$\\
\multicolumn{4}{c}{.\ \ .\ \ .\ \ .\ \ .\ \ .\ \ .}\\
$i=9$&$j=3$&$k=Jun$&$l=b$\\
$i=9$&$j=3$&$k=Jun$&$l=c$\\
\end{tabular}

\medskip

Let the example indexing expression be used in the following iterated
operation:

\medskip

\noindent\hfil
{\tt sum\{i in A, (j,k) in B, l in C\} p[i,j,k,l]}

\medskip

\noindent where {\tt p} is a 4-dimensional numeric parameter or some
numeric expression whose resultant value depends on {\tt i}, {\tt j},
{\tt k}, and {\tt l}. In this case the action is summation, so the
resultant value of the primary numeric expression is:
$$\sum_{i\in A,(j,k)\in B,l\in C}(p_{ijkl}).$$

Now let the example indexing expression be used as a primary set
expression. In this case the action is gathering all 4-tuples
(quadruplets) of the form $(i,j,k,l)$ in one set, so the resultant
value of such operation is simply the Cartesian product of the basic
sets:
$$A\times B\times C=\{(i,j,k,l):i\in A,(j,k)\in B,l\in C\}.$$
Note that in this case the same indexing expression might be written in
the reduced form:

\medskip

\noindent\hfil
{\tt\{A, B, C\}}

\medskip

\noindent because the dummy indices $i$, $j$, $k$, and $l$ are not
referenced and therefore their symbolic names need not be specified.

Finally, let the example indexing expression be used as the subscript
domain in the declaration of a 4-dimensional model object, say,
a numeric parameter:

\medskip

\noindent\hfil
{\tt param p\{i in A, (j,k) in B, l in C\}} \dots {\tt;}

\medskip

\noindent In this case the action is generating the parameter members,
where each member has the form $p[i,j,k,l]$.

As was said above, some indices in the second form of indexing entries
may be numeric or symbolic expressions, not only dummy indices. In this
case resultant values of such expressions play role of some logical
conditions to select only that $n$-tuples from the Cartesian product of
basic sets that satisfy these conditions.

Consider, for example, the following indexing expression:

\medskip

\noindent\hfil
{\tt\{i in A, (i-1,k) in B, l in C\}}

\medskip

\noindent where {\tt i}, {\tt k}, {\tt l} are dummy indices, and
{\tt i-1} is a numeric expression. The algorithmic decsription of this
indexing expression is the following:

\medskip

\noindent\hfil
\begin{tabular}{@{}l@{}}
{\bf for all} $i\in A$ {\bf do}\\
\hspace{12pt}{\bf for all} $(j,k)\in B$ {\bf and} $j=i-1$ {\bf do}\\
\hspace{24pt}{\bf for all} $l\in C$ {\bf do}\\
\hspace{36pt}{\it action};\\
\end{tabular}

\medskip

\noindent Thus, if this indexing expression were used as a primary set
expression, the resultant set would be the following:
$$\{(4,May,a),(4,May,b),(4,May,c),(4,Jun,a),(4,Jun,b),(4,Jun,c)\}.$$
Should note that in this case the resultant set consists of 3-tuples,
not of 4-tuples, because in the indexing expression there is no dummy
index that corresponds to the first component of 2-tuples from the set
$B$.

The general rule is: the number of components of $n$-tuples defined by
an indexing expression is the same as the number of dummy indices in
that expression, where the correspondence between dummy indices and
components on $n$-tuples in the resultant set is positional, i.e. the
first dummy index corresponds to the first component, the second dummy
index corresponds to the second component, etc.

In some cases it is needed to select a subset from the Cartesian
product of some sets. This may be attained by using an optional logical
predicate, which is specified in the indexing expression.

Consider, for example, the following indexing expression:

\medskip

\noindent\hfil
{\tt\{i in A, (j,k) in B, l in C: i <= 5 and k <> 'Mar'\}}

\medskip

\noindent where the logical expression following the colon is a
predicate. The algorithmic description of this indexing expression is
the following:

\medskip

\noindent\hfil
\begin{tabular}{@{}l@{}}
{\bf for all} $i\in A$ {\bf do}\\
\hspace{12pt}{\bf for all} $(j,k)\in B$ {\bf do}\\
\hspace{24pt}{\bf for all} $l\in C$ {\bf do}\\
\hspace{36pt}{\bf if} $i\leq 5$ {\bf and} $l\neq`Mar'$ {\bf then}\\
\hspace{48pt}{\it action};\\
\end{tabular}

\medskip

\noindent Thus, if this indexing expression were used as a primary set
expression, the resultant set would be the following:
$$\{(4,1,Jan,a),(4,1,Feb,a),(4,2,Apr,a),\dots,(4,3,Jun,c)\}.$$

If no predicate is specified in the indexing expression, one, which
takes on the value {\it true}, is assumed.

\subsection{Set expressions}

A {\it set expression} is a rule for computing an elemental set, i.e.
a collection of $n$-tuples, where components of $n$-tuples are numeric
and symbolic quantities.

The primary set expression may be a literal set, unsubscripted set,
subscripted set, ``arithmetic'' set, indexing expression, iterated set
expression, conditional set expression, or another set expression
enclosed in parentheses.

\medskip

\noindent{\bf Examples}

\medskip

\noindent
\begin{tabular}{@{}ll@{}}
\verb|{(123,'aa'), (i,'bb'), (j-1,'cc')}|&(literal set)\\
\verb|I|&(unsubscripted set)\\
\verb|S[i-1,j+1]|&(subscripted set)\\
\verb|1..t-1 by 2|&(``arithmetic'' set)\\
\verb|{t in 1..T, (t+1,j) in S: (t,j) in F}|&(indexing expression)\\
\verb|setof{i in I, j in J}(i+1,j-1)|&(iterated expression)\\
\verb|if i < j then S[i] else F diff S[j]|&(conditional expression)\\
\verb|(1..10 union 21..30)|&(parenthesized expression)\\
\end{tabular}

\medskip

More general set expressions containing two or more primary set
expressions may be constructed by using certain set operators.

\medskip

\noindent{\bf Examples}

\medskip

\noindent\verb|(A union B) inter (I cross J)|

\noindent
\verb|1..10 cross (if i < j then {'a', 'b', 'c'} else {'d', 'e', 'f'})|

\subsubsection{Literal sets}

A {\it literal set} is a primary set expression, which has the
following two syntactic forms:

\medskip

\noindent\hspace{39pt}
{\tt\{}$e_1${\tt,} $e_2${\tt,} \dots{\tt,} $e_m${\tt\}}

\medskip

\noindent\hfil
{\tt\{(}$e_{11}${\tt,} \dots{\tt,} $e_{1n}${\tt),}
{\tt(}$e_{21}${\tt,} \dots{\tt,} $e_{2n}${\tt),} \dots{\tt,}
{\tt(}$e_{m1}${\tt,} \dots{\tt,} $e_{mn}${\tt)\}}

\medskip

\noindent where $e_1$, \dots, $e_m$, $e_{11}$, \dots, $e_{mn}$ are
numeric or symbolic expressions.

If the first form is used, the resultant set consists of 1-tuples
(singlets) enumerated within the curly braces. It is allowed to specify
an empty set as {\tt\{\ \}}, which has no 1-tuples. If the second form
is used, the resultant set consists of $n$-tuples enumerated within the
curly braces, where a particular $n$-tuple consists of corresponding
components enumerated within the parentheses. All $n$-tuples must have
the same number of components.

\subsubsection{Unsubscripted sets}

If the primary set expression is an unsubscripted set (which must be
0-dimen\-sional), the resultant set is an elemental set associated with
the corresponding set object.

\newpage

\subsubsection{Subscripted sets}

The primary set expression, which refers to a subscripted set, has the
following syntactic form:

\medskip

\noindent\hfil
{\it name}{\tt[}$i_1${\tt,} $i_2${\tt,} \dots{\tt,} $i_n${\tt]}

\medskip

\noindent where {\it name} is the symbolic name of the set object,
$i_1$, $i_2$, \dots, $i_n$ are subscripts.

Each subscript must be a numeric or symbolic expression. The number of
subscripts in the subscript list must be the same as the dimension of
the set object with which the subscript list is associated.

Actual values of subscript expressions are used to identify a
particular member of the set object that determines the resultant set.

\subsubsection{``Arithmetic'' sets}

The primary set expression, which is an ``arithmetic'' set, has the
following two syntactic forms:

\medskip

\noindent\hfil
$t_0$ {\tt..} $t_1$ {\tt by} $\delta t$

\medskip

\noindent\hspace{138.5pt}
$t_0$ {\tt..} $t_1$

\medskip

\noindent where $t_0$, $t_1$, and $\delta t$ are numeric expressions
(the value of $\delta t$ must not be zero). The second form is
equivalent to the first form, where $\delta t=1$.

If $\delta t>0$, the resultant set is determined as follows:
$$\{t:\exists k\in{\cal Z}(t=t_0+k\delta t,\ t_0\leq t\leq t_1)\}.$$
Otherwise, if $\delta t<0$, the resultant set is determined as follows:
$$\{t:\exists k\in{\cal Z}(t=t_0+k\delta t,\ t_1\leq t\leq t_0)\}.$$

\subsubsection{Indexing expressions}

If the primary set expression is an indexing expression, the resultant
set is determined as described above in Subsection \ref{indexing}, page
\pageref{indexing}.

\subsubsection{Iterated expressions}

An {\it iterated set expression} is a primary set expression, which has
the following syntactic form:

\medskip

\noindent\hfil
{\tt setof} {\it indexing-expression} {\it integrand}

\medskip

\noindent where {\it indexing-expression} is an indexing expression,
which introduces dummy indices and controls iterating, {\it integrand}
is either a single numeric or symbolic expression or a list of numeric
and symbolic expressions separated by commae and enclosed in
parentheses.

If the integrand is a single numeric or symbolic expression, the
resultant set consists of 1-tuples and is determined as follows:
$$\{x:(i_1,\dots,i_n)\in\Delta\},$$
\noindent where $x$ is a value of the integrand, $i_1$, \dots, $i_n$
are dummy indices introduced in the indexing expression, $\Delta$ is
the domain, a set of $n$-tuples specified by the indexing expression,
which defines particular values assigned to the dummy indices on
performing the iterated operation.

If the integrand is a list containing $m$ numeric and symbolic
expressions, the resultant set consists of $m$-tuples and is determined
as follows:
$$\{(x_1,\dots,x_m):(i_1,\dots,i_n)\in\Delta\},$$
where $x_1$, \dots, $x_m$ are values of the expressions in the
integrand list, $i_1$, \dots, $i_n$ and $\Delta$ have the same meaning
as above.

\subsubsection{Conditional expressions}

A {\it conditional set expression} is a primary set expression that has
the following syntactic form:

\medskip

\noindent\hfil
{\tt if} $b$ {\tt then} $X$ {\tt else} $Y$

\medskip

\noindent where $b$ is an logical expression, $X$ and $Y$ are set
expressions, which must define sets of the same dimension.

The resultant value of the conditional expression depends on the value
of the logical expression that follows the keyword {\tt if}. If it
takes on the value {\it true}, the resultant set is the value of the
expression that follows the keyword {\tt then}. Otherwise, if the
logical expression takes on the value {\it false}, the resultant set is
the value of the expression that follows the keyword {\tt else}.

\subsubsection{Parenthesized expressions}

Any set expression may be enclosed in parentheses that syntactically
makes it a primary set expression.

Parentheses may be used in set expressions, as in algebra, to specify
the desired order in which operations are to be performed. Where
parentheses are used, the expression within the parentheses is
evaluated before the resultant value is used.

The resultant value of the parenthesized expression is the same as the
value of the expression enclosed within parentheses.

\subsubsection{Set operators}

In MathProg there exist the following set operators, which may be used
in set expressions:

\medskip

\begin{tabular}{@{}p{96pt}p{222pt}@{}}
$X$ {\tt union} $Y$&union $X\cup Y$\\
$X$ {\tt diff} $Y$&difference $X\backslash Y$\\
$X$ {\tt symdiff} $Y$&symmetric difference $X\oplus Y$\\
$X$ {\tt inter} $Y$&intersection $X\cap Y$\\
$X$ {\tt cross} $Y$&cross (Cartesian) product $X\times Y$\\
\end{tabular}

\medskip

\noindent where $X$ and Y are set expressions, which must define sets
of the identical dimension (except the Cartesian product).

If the expression includes more than one set operator, all operators
are performed from left to right according to the hierarchy of
operations (see below).

The resultant value of the expression, which contains set operators, is
the result of applying the operators to their operands.

The dimension of the resultant set, i.e. the dimension of $n$-tuples,
of which the resultant set consists of, is the same as the dimension of
the operands, except the Cartesian product, where the dimension of the
resultant set is the sum of the dimensions of its operands.

\subsubsection{Hierarchy of operations}

The following list shows the hierarchy of operations in set
expressions:

\medskip

\noindent\hfil
\begin{tabular}{@{}ll@{}}
Operation&Hierarchy\\
\hline
Evaluation of numeric operations&1st-7th\\
Evaluation of symbolic operations&8th-9th\\
Evaluation of iterated or ``arithmetic'' set ({\tt setof}, {\tt..})&
10th\\
Cartesian product ({\tt cross})&11th\\
Intersection ({\tt inter})&12th\\
Union and difference ({\tt union}, {\tt diff}, {\tt symdiff})&13th\\
Conditional evaluation ({\tt if} \dots {\tt then} \dots {\tt else})&
14th\\
\end{tabular}

\medskip

This hierarchy has the same meaning as was explained above for numeric
expressions (see Subsection \ref{hierarchy}, page \pageref{hierarchy}).

\subsection{Logical expressions}

A {\it logical expression} is a rule for computing a single logical
value, which can be either {\it true} or {\it false}.

The primary logical expression may be a numeric expression, relational
expression, iterated logical expression, or another logical expression
enclosed in parentheses.

\medskip

\noindent{\bf Examples}

\medskip

\noindent
\begin{tabular}{@{}ll@{}}
\verb|i+1|&(numeric expression)\\
\verb|a[i,j] < 1.5|&(relational expression)\\
\verb|s[i+1,j-1] <> 'Mar'|&(relational expression)\\
\verb|(i+1,'Jan') not in I cross J|&(relational expression)\\
\verb|S union T within A[i] inter B[j]|&(relational expression)\\
\verb|forall{i in I, j in J} a[i,j] < .5 * b|&(iterated expression)\\
\verb|(a[i,j] < 1.5 or b[i] >= a[i,j])|&(parenthesized expression)\\
\end{tabular}

\medskip

More general logical expressions containing two or more primary logical
expressions may be constructed by using certain logical operators.

\newpage

\noindent{\bf Examples}

\medskip

\noindent\verb|not (a[i,j] < 1.5 or b[i] >= a[i,j]) and (i,j) in S|

\noindent\verb|(i,j) in S or (i,j) not in T diff U|

\subsubsection{Numeric expressions}

The resultant value of the primary logical expression, which is a
numeric expression, is {\it true}, if the resultant value of the
numeric expression is non-zero. Otherwise the resultant value of the
logical expression is {\it false}.

\subsubsection{Relational operators}

In MathProg there exist the following relational operators, which may
be used in logical expressions:

\medskip

\begin{tabular}{@{}ll@{}}
$x$ {\tt<} $y$&test on $x<y$\\
$x$ {\tt<=} $y$&test on $x\leq y$\\
$x$ {\tt=} $y$, $x$ {\tt==} $y$&test on $x=y$\\
$x$ {\tt>=} $y$&test on $x\geq y$\\
$x$ {\tt>} $y$&test on $x>y$\\
$x$ {\tt<>} $y$, $x$ {\tt!=} $y$&test on $x\neq y$\\
$x$ {\tt in} $Y$&test on $x\in Y$\\
{\tt(}$x_1${\tt,}\dots{\tt,}$x_n${\tt)} {\tt in} $Y$&test on
$(x_1,\dots,x_n)\in Y$\\
$x$ {\tt not} {\tt in} $Y$, $x$ {\tt!in} $Y$&test on $x\not\in Y$\\
{\tt(}$x_1${\tt,}\dots{\tt,}$x_n${\tt)} {\tt not} {\tt in} $Y$,
{\tt(}$x_1${\tt,}\dots{\tt,}$x_n${\tt)} {\tt !in} $Y$&test on
$(x_1,\dots,x_n)\not\in Y$\\
$X$ {\tt within} $Y$&test on $X\subseteq Y$\\
$X$ {\tt not} {\tt within} $Y$, $X$ {\tt !within} $Y$&test on
$X\not\subseteq Y$\\
\end{tabular}

\medskip

\noindent where $x$, $x_1$, \dots, $x_n$, $y$ are numeric or symbolic
expressions, $X$ and $Y$ are set expression.

{\it Notes:}

1. In the operations {\tt in}, {\tt not in}, and {\tt !in} the
number of components in the first operands must be the same as the
dimension of the second operand.

2. In the operations {\tt within}, {\tt not within}, and {\tt !within}
both operands must have identical dimension.

All the relational operators listed above have their conventional
mathematical meaning. The resultant value is {\it true}, if
corresponding relation is satisfied for its operands, otherwise
{\it false}. (Note that symbolic values are ordered lexicographically,
and any numeric value precedes any symbolic value.)

\subsubsection{Iterated expressions}

An {\it iterated logical expression} is a primary logical expression,
which has the following syntactic form:

\medskip

\noindent\hfil
{\it iterated-operator} {\it indexing-expression} {\it integrand}

\medskip

\noindent where {\it iterated-operator} is the symbolic name of the
iterated operator to be performed (see below), {\it indexing-expression}
is an indexing expression which introduces dummy indices and controls
iterating, {\it integrand} is a numeric expression that participates in
the operation.

In MathProg there exist two iterated operators, which may be used in
logical expressions:

\medskip

\noindent\hfil
\begin{tabular}{@{}lll@{}}
{\tt forall}&$\forall$-quantification&$\displaystyle
\forall(i_1,\dots,i_n)\in\Delta[f(i_1,\dots,i_n)],$\\
{\tt exists}&$\exists$-quantification&$\displaystyle
\exists(i_1,\dots,i_n)\in\Delta[f(i_1,\dots,i_n)],$\\
\end{tabular}

\medskip

\noindent where $i_1$, \dots, $i_n$ are dummy indices introduced in
the indexing expression, $\Delta$ is the domain, a set of $n$-tuples
specified by the indexing expression which defines particular values
assigned to the dummy indices on performing the iterated operation,
$f(i_1,\dots,i_n)$ is the integrand, a logical expression whose
resultant value depends on the dummy indices.

For $\forall$-quantification the resultant value of the iterated
logical expression is {\it true}, if the value of the integrand is
{\it true} for all $n$-tuples contained in the domain, otherwise
{\it false}.

For $\exists$-quantification the resultant value of the iterated
logical expression is {\it false}, if the value of the integrand is
{\it false} for all $n$-tuples contained in the domain, otherwise
{\it true}.

\subsubsection{Parenthesized expressions}

Any logical expression may be enclosed in parentheses that
syntactically makes it a primary logical expression.

Parentheses may be used in logical expressions, as in algebra, to
specify the desired order in which operations are to be performed.
Where parentheses are used, the expression within the parentheses is
evaluated before the resultant value is used.

The resultant value of the parenthesized expression is the same as the
value of the expression enclosed within parentheses.

\subsubsection{Logical operators}

In MathProg there exist the following logical operators, which may be
used in logical expressions:

\medskip

\begin{tabular}{@{}p{96pt}p{222pt}@{}}
{\tt not} $x$, {\tt!}$x$&negation $\neg\ x$\\
$x$ {\tt and} $y$, $x$ {\tt\&\&} $y$&conjunction (logical ``and'')
$x\;\&\;y$\\
$x$ {\tt or} $y$, $x$ {\tt||} $y$&disjunction (logical ``or'')
$x\vee y$\\
\end{tabular}

\medskip

\noindent where $x$ and $y$ are logical expressions.

If the expression includes more than one logical operator, all
operators are performed from left to right according to the hierarchy
of the operations (see below). The resultant value of the expression,
which contains logical operators, is the result of applying the
operators to their operands.

\subsubsection{Hierarchy of operations}

The following list shows the hierarchy of operations in logical
expressions:

\medskip

\noindent\hfil
\begin{tabular}{@{}ll@{}}
Operation&Hierarchy\\
\hline
Evaluation of numeric operations&1st-7th\\
Evaluation of symbolic operations&8th-9th\\
Evaluation of set operations&10th-14th\\
Relational operations ({\tt<}, {\tt<=}, etc.)&15th\\
Negation ({\tt not}, {\tt!})&16th\\
Conjunction ({\tt and}, {\tt\&\&})&17th\\
$\forall$- and $\exists$-quantification ({\tt forall}, {\tt exists})&
18th\\
Disjunction ({\tt or}, {\tt||})&19th\\
\end{tabular}

\medskip

This hierarchy has the same meaning as was explained above for numeric
expressions (see Subsection \ref{hierarchy}, page \pageref{hierarchy}).

\subsection{Linear expressions}

An {\it linear expression} is a rule for computing so called
a {\it linear form} or simply a {\it formula}, which is a linear (or
affine) function of elemental variables.

The primary linear expression may be an unsubscripted variable,
subscripted variable, iterated linear expression, conditional linear
expression, or another linear expression enclosed in parentheses.

It is also allowed to use a numeric expression as the primary linear
expression, in which case the resultant value of the numeric expression
is automatically converted to a formula that includes the constant term
only.

\medskip

\noindent{\bf Examples}

\medskip

\noindent
\begin{tabular}{@{}ll@{}}
\verb|z|&(unsubscripted variable)\\
\verb|x[i,j]|&(subscripted variable)\\
\verb|sum{j in J} (a[i] * x[i,j] + 3 * y)|&(iterated expression)\\
\verb|if i in I then x[i,j] else 1.5 * z + 3|&(conditional expression)\\
\verb|(a[i,j] * x[i,j] + y[i-1] + .1)|&(parenthesized expression)\\
\end{tabular}

\medskip

More general linear expressions containing two or more primary linear
expressions may be constructed by using certain arithmetic operators.

\medskip

\noindent{\bf Examples}

\medskip

\noindent\verb|2 * x[i-1,j+1] + 3.5 * y[k] + .5 * z|

\noindent\verb|(- x[i,j] + 3.5 * y[k]) / sum{t in T} abs(d[i,j,t])|

\subsubsection{Unsubscripted variables}

If the primary linear expression is an unsubscripted variable (which
must be 0-dimensional), the resultant formula is that unsubscripted
variable.

\subsubsection{Subscripted variables}

The primary linear expression, which refers to a subscripted variable,
has the following syntactic form:

\medskip

\noindent\hfil
{\it name}{\tt[}$i_1${\tt,} $i_2${\tt,} \dots{\tt,} $i_n${\tt]}

\medskip

\noindent where {\it name} is the symbolic name of the model variable,
$i_1$, $i_2$, \dots, $i_n$ are subscripts.

Each subscript must be a numeric or symbolic expression. The number of
subscripts in the subscript list must be the same as the dimension of
the model variable with which the subscript list is associated.

Actual values of the subscript expressions are used to identify a
particular member of the model variable that determines the resultant
formula, which is an elemental variable associated with corresponding
member.

\subsubsection{Iterated expressions}

An {\it iterated linear expression} is a primary linear expression,
which has the following syntactic form:

\medskip

\noindent\hfil
{\tt sum} {\it indexing-expression} {\it integrand}

\medskip

\noindent where {\it indexing-expression} is an indexing expression,
which introduces dummy indices and controls iterating, {\it integrand}
is a linear expression that participates in the operation.

The iterated linear expression is evaluated exactly in the same way as
the iterated numeric expression (see Subection \ref{itexpr}, page
\pageref{itexpr}) with exception that the integrand participated in the
summation is a formula, not a numeric value.

\subsubsection{Conditional expressions}

A {\it conditional linear expression} is a primary linear expression,
which has one of the following two syntactic forms:

\medskip

\noindent\hfil
{\tt if} $b$ {\tt then} $f$ {\tt else} $g$

\medskip

\noindent\hspace{127pt}
{\tt if} $b$ {\tt then} $f$

\medskip

\noindent where $b$ is an logical expression, $f$ and $g$ are linear
expressions.

The conditional linear expression is evaluated exactly in the same way
as the conditional numeric expression (see Subsection \ref{ifthen},
page \pageref{ifthen}) with exception that operands participated in the
operation are formulae, not numeric values.

\subsubsection{Parenthesized expressions}

Any linear expression may be enclosed in parentheses that syntactically
makes it a primary linear expression.

Parentheses may be used in linear expressions, as in algebra, to
specify the desired order in which operations are to be performed.
Where parentheses are used, the expression within the parentheses is
evaluated before the resultant formula is used.

The resultant value of the parenthesized expression is the same as the
value of the expression enclosed within parentheses.

\subsubsection{Arithmetic operators}

In MathProg there exists the following arithmetic operators, which may
be used in linear expressions:

\medskip

\begin{tabular}{@{}p{96pt}p{222pt}@{}}
{\tt+} $f$&unary plus\\
{\tt-} $f$&unary minus\\
$f$ {\tt+} $g$&addition\\
$f$ {\tt-} $g$&subtraction\\
$x$ {\tt*} $f$, $f$ {\tt*} $x$&multiplication\\
$f$ {\tt/} $x$&division
\end{tabular}

\medskip

\noindent where $f$ and $g$ are linear expressions, $x$ is a numeric
expression (more precisely, a linear expression containing only the
constant term).

If the expression includes more than one arithmetic operator, all
operators are performed from left to right according to the hierarchy
of operations (see below). The resultant value of the expression, which
contains arithmetic operators, is the result of applying the operators
to their operands.

\subsubsection{Hierarchy of operations}

The hierarchy of arithmetic operations used in linear expressions is
the same as for numeric expressions (see Subsection \ref{hierarchy},
page \pageref{hierarchy}).

%%%%%%%%%%%%%%%%%%%%%%%%%%%%%%%%%%%%%%%%%%%%%%%%%%%%%%%%%%%%%%%%%%%%%%%%

\newpage

\section{Statements}

{\it Statements} are basic units of the model description. In MathProg
all statements are divided into two categories: declaration statements
and functional statements.

{\it Declaration statements} (set statement, parameter statement,
variable statement, constraint statement, and objective statement) are
used to declare model objects of certain kinds and define certain
properties of such objects.

{\it Functional statements} (solve statement, check statement, display
statement, printf statement, loop statement) are intended for
performing some specific actions.

Note that declaration statements may follow in arbitrary order, which
does not affect the result of translation. However, any model object
must be declared before it is referenced in other statements.

\subsection{Set statement}

\medskip

\framebox[345pt][l]{
\parbox[c][24pt]{345pt}{
\hspace{6pt} {\tt set} {\it name} {\it alias} {\it domain} {\tt,}
{\it attrib} {\tt,} \dots {\tt,} {\it attrib} {\tt;}
}}

\setlength{\leftmargini}{60pt}

\begin{description}
\item[{\rm Where:}\hspace*{23pt}] {\it name} is a symbolic name of the
set;
\item[\hspace*{54pt}] {\it alias} is an optional string literal, which
specifies an alias of the set;
\item[\hspace*{54pt}] {\it domain} is an optional indexing expression,
which specifies a subscript domain of the set;
\item[\hspace*{54pt}] {\it attrib}, \dots, {\it attrib} are optional
attributes of the set. (Commae preceding attributes may be omitted.)
\end{description}

\noindent Optional attributes:

\begin{description}
\item[{\tt dimen} $n$\hspace*{19pt}] specifies the dimension of
$n$-tuples, which the set consists of;
\item[{\tt within} {\it expression}]\hspace*{0pt}\\
specifies a superset which restricts the set or all its members
(elemental sets) to be within that superset;
\item[{\tt:=} {\it expression}]\hspace*{0pt}\\
specifies an elemental set assigned to the set or its members;
\item[{\tt default} {\it expression}]\hspace*{0pt}\\
specifies an elemental set assigned to the set or its members whenever
no appropriate data are available in the data section.
\end{description}

\newpage

\noindent{\bf Examples}

\begin{verbatim}
set V;
set E within V cross V;
set step{s in 1..maxiter} dimen 2 := if s = 1 then E else
   step[s-1] union setof{k in V, (i,k) in step[s-1], (k,j)
   in step[s-1]}(i,j);
set A{i in I, j in J}, within B[i+1] cross C[j-1], within
   D diff E, default {('abc',123), (321,'cba')};
\end{verbatim}

The set statement declares a set. If the subscript domain is not
specified, the set is a simple set, otherwise it is an array of
elemental sets.

The {\tt dimen} attribute specifies the dimension of $n$-tuples, which
the set (if it is a simple set) or its members (if the set is an array
of elemental sets) consist of, where $n$ must be unsigned integer from
1 to 20. At most one {\tt dimen} attribute can be specified. If the
{\tt dimen} attribute is not specified, the dimension of\linebreak
$n$-tuples is implicitly determined by other attributes (for example,
if there is a set expression that follows {\tt:=} or the keyword
{\tt default}, the dimension of $n$-tuples of corresponding elemental
set is used). If no dimension information is available, {\tt dimen 1}
is assumed.

The {\tt within} attribute specifies a set expression whose resultant
value is a superset used to restrict the set (if it is a simple set) or
its members (if the set is an array of elemental sets) to be within
that superset. Arbitrary number of {\tt within} attributes may be
specified in the same set statement.

The assign ({\tt:=}) attribute specifies a set expression used to
evaluate elemental set(s) assigned to the set (if it is a simple set)
or its members (if the set is an array of elemental sets). If the
assign attribute is specified, the set is {\it computable} and
therefore needs no data to be provided in the data section. If the
assign attribute is not specified, the set must be provided with data
in the data section. At most one assign or default attribute can be
specified for the same set.

The {\tt default} attribute specifies a set expression used to evaluate
elemental set(s) assigned to the set (if it is a simple set) or its
members (if the set is an array of elemental sets) whenever
no appropriate data are available in the data section. If neither
assign nor default attribute is specified, missing data will cause an
error.

\subsection{Parameter statement}

\medskip

\framebox[345pt][l]{
\parbox[c][24pt]{345pt}{
\hspace{6pt} {\tt param} {\it name} {\it alias} {\it domain} {\tt,}
{\it attrib} {\tt,} \dots {\tt,} {\it attrib} {\tt;}
}}

\setlength{\leftmargini}{60pt}

\begin{description}
\item[{\rm Where:}\hspace*{23pt}] {\it name} is a symbolic name of the
parameter;
\item[\hspace*{54pt}] {\it alias} is an optional string literal, which
specifies an alias of the parameter;
\item[\hspace*{54pt}] {\it domain} is an optional indexing expression,
which specifies a subscript domain of the parameter;
\item[\hspace*{54pt}] {\it attrib}, \dots, {\it attrib} are optional
attributes of the parameter. (Commae preceding attributes may be
omitted.)
\end{description}

\noindent Optional attributes:

\begin{description}
\item[{\tt integer}\hspace*{18.5pt}] specifies that the parameter is
integer;
\item[{\tt binary}\hspace*{24pt}] specifies that the parameter is
binary;
\item[{\tt symbolic}\hspace*{13.5pt}] specifies that the parameter is
symbolic;
\item[{\it relation expression}]\hspace*{0pt}\\
(where {\it relation} is one of: {\tt<}, {\tt<=}, {\tt=}, {\tt==},
{\tt>=}, {\tt>}, {\tt<>}, {\tt!=})\\
specifies a condition that restricts the parameter or its members to
satisfy that condition;
\item[{\tt in} {\it expression}]\hspace*{0pt}\\
specifies a superset that restricts the parameter or its members to be
in that superset;
\item[{\tt:=} {\it expression}]\hspace*{0pt}\\
specifies a value assigned to the parameter or its members;
\item[{\tt default} {\it expression}]\hspace*{0pt}\\
specifies a value assigned to the parameter or its members whenever
no appropriate data are available in the data section.
\end{description}

\noindent{\bf Examples}

\begin{verbatim}
param units{raw, prd} >= 0;
param profit{prd, 1..T+1};
param N := 20, integer, >= 0, <= 100;
param comb 'n choose k' {n in 0..N, k in 0..n} :=
   if k = 0 or k = n then 1 else comb[n-1,k-1] + comb[n-1,k];
param p{i in I, j in J}, integer, >= 0, <= i+j,
   in A[i] symdiff B[j], in C[i,j], default 0.5 * (i + j);
param month symbolic default 'May' in {'Mar', 'Apr', 'May'};
\end{verbatim}

The parameter statement declares a parameter. If a subscript domain is
not specified, the parameter is a simple (scalar) parameter, otherwise
it is a $n$-dimensional array.

The type attributes {\tt integer}, {\tt binary}, and {\tt symbolic}
qualify the type of values that can be assigned to the parameter as
shown below:

\medskip

\noindent\hfil
\begin{tabular}{@{}ll@{}}
Type attribute&Assigned values\\
\hline
(not specified)&Any numeric values\\
{\tt integer}&Only integer numeric values\\
{\tt binary}&Either 0 or 1\\
{\tt symbolic}&Any numeric and symbolic values\\
\end{tabular}

\newpage

The {\tt symbolic} attribute cannot be specified along with other type
attributes. Being specified it must precede all other attributes.

The condition attribute specifies an optional condition that restricts
values assigned to the parameter to satisfy that condition. This
attribute has the following syntactic forms:

\medskip

\begin{tabular}{@{}ll@{}}
{\tt<} $v$&check for $x<v$\\
{\tt<=} $v$&check for $x\leq v$\\
{\tt=} $v$, {\tt==} $v$&check for $x=v$\\
{\tt>=} $v$&check for $x\geq v$\\
{\tt>} $v$&check for $x\geq v$\\
{\tt<>} $v$, {\tt!=} $v$&check for $x\neq v$\\
\end{tabular}

\medskip

\noindent where $x$ is a value assigned to the parameter, $v$ is the
resultant value of a numeric or symbolic expression specified in the
condition attribute. Arbitrary number of condition attributes can be
specified for the same parameter. If a value being assigned to the
parameter during model evaluation violates at least one of specified
conditions, an error is raised. (Note that symbolic values are ordered
lexicographically, and any numeric value precedes any symbolic value.)

The {\tt in} attribute is similar to the condition attribute and
specifies a set expression whose resultant value is a superset used to
restrict numeric or symbolic values assigned to the parameter to be in
that superset. Arbitrary number of the {\tt in} attributes can be
specified for the same parameter. If a value being assigned to the
parameter during model evaluation is not in at least one of specified
supersets, an error is raised.

The assign ({\tt:=}) attribute specifies a numeric or symbolic
expression used to compute a value assigned to the parameter (if it is
a simple parameter) or its member (if the parameter is an array). If
the assign attribute is specified, the parameter is {\it computable}
and therefore needs no data to be provided in the data section. If the
assign attribute is not specified, the parameter must be provided with
data in the data section. At most one assign or {\tt default} attribute
can be specified for the same parameter.

The {\tt default} attribute specifies a numeric or symbolic expression
used to compute a value assigned to the parameter or its member
whenever no appropriate data are available in the data section. If
neither assign nor {\tt default} attribute is specified, missing data
will cause an error.

\subsection{Variable statement}

\medskip

\framebox[345pt][l]{
\parbox[c][24pt]{345pt}{
\hspace{6pt} {\tt var} {\it name} {\it alias} {\it domain} {\tt,}
{\it attrib} {\tt,} \dots {\tt,} {\it attrib} {\tt;}
}}

\setlength{\leftmargini}{60pt}

\begin{description}
\item[{\rm Where:}\hspace*{23pt}] {\it name} is a symbolic name of the
variable;
\item[\hspace*{54pt}] {\it alias} is an optional string literal, which
specifies an alias of the variable;
\item[\hspace*{54pt}] {\it domain} is an optional indexing expression,
which specifies a subscript domain of the variable;
\item[\hspace*{54pt}] {\it attrib}, \dots, {\it attrib} are optional
attributes of the variable. (Commae preceding attributes may be
omitted.)
\end{description}

\noindent Optional attributes:

\begin{description}
\item[{\tt integer}\hspace*{18.5pt}] restricts the variable to be
integer;
\item[{\tt binary}\hspace*{24pt}] restricts the variable to be binary;
\item[{\tt>=} {\it expression}]\hspace*{0pt}\\
specifies an lower bound of the variable;
\item[{\tt<=} {\it expression}]\hspace*{0pt}\\
specifies an upper bound of the variable;
\item[{\tt=} {\it expression}]\hspace*{0pt}\\
specifies a fixed value of the variable;
\end{description}

\noindent{\bf Examples}

\begin{verbatim}
var x >= 0;
var y{I,J};
var make{p in prd}, integer, >= commit[p], <= market[p];
var store{raw, 1..T+1} >= 0;
var z{i in I, j in J} >= i+j;
\end{verbatim}

The variable statement declares a variable. If a subscript domain is
not specified, the variable is a simple (scalar) variable, otherwise it
is a $n$-dimensional array of elemental variables.

Elemental variable(s) associated with the model variable (if it is a
simple variable) or its members (if it is an array) correspond to the
variables in the LP/MIP problem formulation (see Subsection
\ref{problem}, page \pageref{problem}). Note that only elemental
variables actually referenced in some constraints and/or objectives are
included in the LP/MIP problem instance to be generated.

The type attributes {\tt integer} and {\tt binary} restrict the
variable to be integer or binary, respectively. If no type attribute is
specified, the variable is continuous. If all variables in the model
are continuous, the corresponding problem is of LP class. If there is
at least one integer or binary variable, the problem is of MIP class.

The lower bound ({\tt>=}) attribute specifies a numeric expression for
computing an lower bound of the variable. At most one lower bound can
be specified. By default all variables (except binary ones) have no
lower bound, so if a variable is required to be non-negative, its zero
lower bound should be explicitly specified.

The upper bound ({\tt<=}) attribute specifies a numeric expression for
computing an upper bound of the variable. At most one upper bound
attribute can be specified.

The fixed value ({\tt=}) attribute specifies a numeric expression for
computing a value, at which the variable is fixed. This attribute
cannot be specified along with the bound attributes.

\subsection{Constraint statement}

\medskip

\framebox[345pt][l]{
\parbox[c][96pt]{345pt}{
\hspace{6pt} {\tt s.t.} {\it name} {\it alias} {\it domain} {\tt:}
{\it expression} {\tt,} {\tt=} {\it expression} {\tt;}

\medskip

\hspace{6pt} {\tt s.t.} {\it name} {\it alias} {\it domain} {\tt:}
{\it expression} {\tt,} {\tt<=} {\it expression} {\tt;}

\medskip

\hspace{6pt} {\tt s.t.} {\it name} {\it alias} {\it domain} {\tt:}
{\it expression} {\tt,} {\tt>=} {\it expression} {\tt;}

\medskip

\hspace{6pt} {\tt s.t.} {\it name} {\it alias} {\it domain} {\tt:}
{\it expression} {\tt,} {\tt<=} {\it expression} {\tt,} {\tt<=}
{\it expression} {\tt;}

\medskip

\hspace{6pt} {\tt s.t.} {\it name} {\it alias} {\it domain} {\tt:}
{\it expression} {\tt,} {\tt>=} {\it expression} {\tt,} {\tt>=}
{\it expression} {\tt;}
}}

\setlength{\leftmargini}{60pt}

\begin{description}
\item[{\rm Where:}\hspace*{23pt}] {\it name} is a symbolic name of the
constraint;
\item[\hspace*{54pt}] {\it alias} is an optional string literal, which
specifies an alias of the constraint;
\item[\hspace*{54pt}] {\it domain} is an optional indexing expression,
which specifies a subscript domain of the constraint;
\item[\hspace*{54pt}] {\it expression} is a linear expression used to
compute a component of the constraint. (Commae following expressions
may be omitted.)
\end{description}

\begin{description}
\item[{\rm Note:}\hspace*{31pt}] The keyword {\tt s.t.} may be written
as {\tt subject to} or as {\tt subj to}, or may be omitted at all.
\end{description}

\noindent{\bf Examples}

\begin{verbatim}
s.t. r: x + y + z, >= 0, <= 1;
limit{t in 1..T}: sum{j in prd} make[j,t] <= max_prd;
subject to balance{i in raw, t in 1..T}: store[i,t+1] -
   store[i,t] - sum{j in prd} units[i,j] * make[j,t];
subject to rlim 'regular-time limit' {t in time}:
sum{p in prd} pt[p] * rprd[p,t] <= 1.3 * dpp[t] * crews[t];
\end{verbatim}

The constraint statement declares a constraint. If a subscript domain
is not specified, the constraint is a simple (scalar) constraint,
otherwise it is a $n$-dimensional array of elemental constraints.

Elemental constraint(s) associated with the model constraint (if it is
a simple constraint) or its members (if it is an array) correspond to
the linear constraints in the LP/MIP problem formulation (see
Subsection \ref{problem}, page \pageref{problem}).

If the constraint has the form of equality or single inequality, i.e.
includes two expressions, one of which follows the colon and other
follows the relation sign {\tt=}, {\tt<=}, or {\tt>=}, both expressions
in the statement can be linear expressions. If the constraint has the
form of double inequality, i.e. includes three expressions, the middle
expression can be a linear expression while the leftmost and rightmost
ones can be only numeric expressions.

Generating the model is, roughly speaking, generating its constraints,
which are always evaluated for the entire subscript domain. Evaluation
of the constraints leads, in turn, to evaluation of other model objects
such as sets, parameters, and variables.

Constructing an actual linear constraint included in the problem
instance, which (constraint) corresponds to a particular elemental
constraint, is performed as follows.

If the constraint has the form of equality or single inequality,
evaluation of both linear expressions gives two resultant linear forms:
$$\begin{array}{r@{\ }c@{\ }r@{\ }c@{\ }r@{\ }c@{\ }r@{\ }c@{\ }r}
f&=&a_1x_1&+&a_2x_2&+\dots+&a_nx_n&+&a_0,\\
g&=&b_1x_1&+&a_2x_2&+\dots+&a_nx_n&+&b_0,\\
\end{array}$$
where $x_1$, $x_2$, \dots, $x_n$ are elemental variables; $a_1$, $a_2$,
\dots, $a_n$, $b_1$, $b_2$, \dots, $b_n$ are numeric coefficients;
$a_0$ and $b_0$ are constant terms. Then all linear terms of $f$ and
$g$ are carried to the left-hand side, and the constant terms are
carried to the right-hand side, that gives the final elemental
constraint in the standard form:
$$(a_1-b_1)x_1+(a_2-b_2)x_2+\dots+(a_n-b_n)x_n\left\{
\begin{array}{@{}c@{}}=\\\leq\\\geq\\\end{array}\right\}b_0-a_0.$$

If the constraint has the form of double inequality, evaluation of the
middle linear expression gives the resultant linear form:
$$f=a_1x_1+a_2x_2+\dots+a_nx_n+a_0,$$
and evaluation of the leftmost and rightmost numeric expressions gives
two numeric values $l$ and $u$, respectively. Then the constant term of
the linear form is carried to both left-hand and right-handsides that
gives the final elemental constraint in the standard form:
$$l-a_0\leq a_1x_1+a_2x_2+\dots+a_nx_n\leq u-a_0.$$

\subsection{Objective statement}

\medskip

\framebox[345pt][l]{
\parbox[c][44pt]{345pt}{
\hspace{6pt} {\tt minimize} {\it name} {\it alias} {\it domain} {\tt:}
{\it expression} {\tt;}

\medskip

\hspace{6pt} {\tt maximize} {\it name} {\it alias} {\it domain} {\tt:}
{\it expression} {\tt;}
}}

\setlength{\leftmargini}{60pt}

\begin{description}
\item[{\rm Where:}\hspace*{23pt}] {\it name} is a symbolic name of the
objective;
\item[\hspace*{54pt}] {\it alias} is an optional string literal, which
specifies an alias of the objective;
\item[\hspace*{54pt}] {\it domain} is an optional indexing expression,
which specifies a subscript domain of the objective;
\item[\hspace*{54pt}] {\it expression} is a linear expression used to
compute the linear form of the objective.
\end{description}

\noindent{\bf Examples}

\begin{verbatim}
minimize obj: x + 1.5 * (y + z);
maximize total_profit: sum{p in prd} profit[p] * make[p];
\end{verbatim}

The objective statement declares an objective. If a subscript domain is
not specified, the objective is a simple (scalar) objective. Otherwise
it is a $n$-dimensional array of elemental objectives.

Elemental objective(s) associated with the model objective (if it is a
simple objective) or its members (if it is an array) correspond to
general linear constraints in the LP/MIP problem formulation (see
Subsection \ref{problem}, page \pageref{problem}). However, unlike
constraints the corresponding linear forms are free (unbounded).

Constructing an actual linear constraint included in the problem
instance, which (constraint) corresponds to a particular elemental
constraint, is performed as follows. The linear expression specified in
the objective statement is evaluated that, gives the resultant linear
form:
$$f=a_1x_1+a_2x_2+\dots+a_nx_n+a_0,$$
where $x_1$, $x_2$, \dots, $x_n$ are elemental variables; $a_1$, $a_2$,
\dots, $a_n$ are numeric coefficients; $a_0$ is the constant term. Then
the linear form is used to construct the final elemental constraint in
the standard form:
$$-\infty<a_1x_1+a_2x_2+\dots+a_nx_n+a_0<+\infty.$$

As a rule the model description contains only one objective statement
that defines the objective function used in the problem instance.
However, it is allowed to declare arbitrary number of objectives, in
which case the actual objective function is the first objective
encountered in the model description. Other objectives are also
included in the problem instance, but they do not affect the objective
function.

\subsection{Solve statement}

\medskip

\framebox[345pt][l]{
\parbox[c][24pt]{345pt}{
\hspace{6pt} {\tt solve} {\tt;}
}}

\setlength{\leftmargini}{60pt}

\begin{description}
\item[{\rm Note:}\hspace*{31pt}] The solve statement is optional and
can be used only once. If no solve statement is used, one is assumed at
the end of the model section.
\end{description}

The solve statement causes the model to be solved, that means computing
numeric values of all model variables. This allows using variables in
statements below the solve statement in the same way as if they were
numeric parameters.

Note that the variable, constraint, and objective statements cannot be
used below the solve statement, i.e. all principal components of the
model must be declared above the solve statement.

\subsection{Check statement}

\medskip

\framebox[345pt][l]{
\parbox[c][24pt]{345pt}{
\hspace{6pt} {\tt check} {\it domain} {\tt:} {\it expression} {\tt;}
}}

\setlength{\leftmargini}{60pt}

\begin{description}
\item[{\rm Where:}\hspace*{23pt}] {\it domain} is an optional indexing
expression, which specifies a subscript domain of the check statement;
\item[\hspace*{54pt}] {\it expression} is an logical expression which
specifies the logical condition to be checked. (The colon preceding
{\it expression} may be omitted.)
\end{description}

\noindent{\bf Examples}

\begin{verbatim}
check: x + y <= 1 and x >= 0 and y >= 0;
check sum{i in ORIG} supply[i] = sum{j in DEST} demand[j];
check{i in I, j in 1..10}: S[i,j] in U[i] union V[j];
\end{verbatim}

The check statement allows checking the resultant value of an logical
expression specified in the statement. If the value is {\it false}, an
error is reported.

If the subscript domain is not specified, the check is performed only
once. Specifying the subscript domain allows performing multiple checks
for every\linebreak $n$-tuple in the domain set. In the latter case the
logical expression may include dummy indices introduced in
corresponding indexing expression.

\subsection{Display statement}

\medskip

\framebox[345pt][l]{
\parbox[c][24pt]{345pt}{
\hspace{6pt} {\tt display} {\it domain} {\tt:} {\it item} {\tt,}
\dots {\tt,} {\it item} {\tt;}
}}

\setlength{\leftmargini}{60pt}

\begin{description}
\item[{\rm Where:}\hspace*{23pt}] {\it domain} is an optional indexing
expression, which specifies a subscript domain of the check statement;
\item[\hspace*{54pt}] {\it item}, \dots, {\it item} are items to be
displayed. (The colon preceding the first item may be omitted.)
\end{description}

\noindent{\bf Examples}

\begin{verbatim}
display: 'x =', x, 'y =', y, 'z =', z;
display sqrt(x ** 2 + y ** 2 + z ** 2);
display{i in I, j in J}: i, j, a[i,j], b[i,j];
\end{verbatim}

\newpage

The display statement evaluates all items specified in the statement
and writes their values to the terminal in plain text format.

If a subscript domain is not specified, items are evaluated and then
displayed only once. Specifying the subscript domain causes items to be
evaluated and displayed for every $n$-tuple in the domain set. In the
latter case items may include dummy indices introduced in corresponding
indexing expression.

An item to be displayed can be a model object (set, parameter, variable,
constraint, objective) or an expression.

If the item is a computable object (i.e. a set or parameter provided
with the assign attribute), the object is evaluated over the entire
domain and then its content (i.e. the content of the object array) is
displayed. Otherwise, if the item is not a computable object, only its
current content (i.e. members actually generated during the model
evaluation) is displayed.

If the item is an expression, the expression is evaluated and its
resultant value is displayed.

\subsection{Printf statement}

\medskip

\framebox[345pt][l]{
\parbox[c][60pt]{345pt}{
\hspace{6pt} {\tt printf} {\it domain} {\tt:} {\it format} {\tt,}
{\it expression} {\tt,} \dots {\tt,} {\it expression} {\tt;}

\medskip

\hspace{6pt} {\tt printf} {\it domain} {\tt:} {\it format} {\tt,}
{\it expression} {\tt,} \dots {\tt,} {\it expression} {\tt>}
{\it filename} {\tt;}

\medskip

\hspace{6pt} {\tt printf} {\it domain} {\tt:} {\it format} {\tt,}
{\it expression} {\tt,} \dots {\tt,} {\it expression} {\tt>>}
{\it filename} {\tt;}
}}

\setlength{\leftmargini}{60pt}

\begin{description}
\item[{\rm Where:}\hspace*{23pt}] {\it domain} is an optional indexing
expression, which specifies a subscript domain of the printf statement;
\item[\hspace*{54pt}] {\it format} is a symbolic expression whose value
specifies a format control string. (The colon preceding the format
expression may be omitted.)
\item[\hspace*{54pt}] {\it expression}, \dots, {\it expression} are
zero or more expressions whose values have to be formatted and printed.
Each expression must be of numeric, symbolic, or logical type.
\item[\hspace*{54pt}] {\it filename} is a symbolic expression whose
value specifies a name of a text file, to which the output is
redirected. The flag {\tt>} means creating a new empty file while the
flag {\tt>>} means appending the output to an existing file. If no file
name is specified, the output is written to the terminal.
\end{description}

\noindent{\bf Examples}

\begin{verbatim}
printf 'Hello, world!\n';
printf: "x = %.3f; y = %.3f; z = %.3f\n",
   x, y, z > "result.txt";
printf{i in I, j in J}: "flow from %s to %s is %d\n",
   i, j, x[i,j] >> result_file & ".txt";
\end{verbatim}

\newpage

\begin{verbatim}
printf{i in I} 'total flow from %s is %g\n',
   i, sum{j in J} x[i,j];
printf{k in K} "x[%s] = " & (if x[k] < 0 then "?" else "%g"),
   k, x[k];
\end{verbatim}

The printf statement is similar to the display statement, however, it
allows formatting data to be written.

If a subscript domain is not specified, the printf statement is
executed only once. Specifying a subscript domain causes executing the
printf statement for every $n$-tuple in the domain set. In the latter
case the format and expression may include dummy indices introduced in
corresponding indexing expression.

The format control string is a value of the symbolic expression
{\it format} specified in the printf statement. It is composed of zero
or more directives as follows: ordinary characters (not {\tt\%}), which
are copied unchanged to the output stream, and conversion
specifications, each of which causes evaluating corresponding
expression specified in the printf statement, formatting it, and
writing its resultant value to the output stream.

Conversion specifications that may be used in the format control string
are the following: {\tt d}, {\tt i}, {\tt f}, {\tt F}, {\tt e}, {\tt E},
{\tt g}, {\tt G}, and {\tt s}. These specifications have the same
syntax and semantics as in the C programming language.

\subsection{For statement}

\medskip

\framebox[345pt][l]{
\parbox[c][44pt]{345pt}{
\hspace{6pt} {\tt for} {\it domain} {\tt:} {\it statement} {\tt;}

\medskip

\hspace{6pt} {\tt for} {\it domain} {\tt:} {\tt\{} {\it statement}
\dots {\it statement} {\tt\}} {\tt;}
}}

\setlength{\leftmargini}{60pt}

\begin{description}
\item[{\rm Where:}\hspace*{23pt}] {\it domain} is an indexing
expression which specifies a subscript domain of the for statement.
(The colon following the indexing expression may be omitted.)
\item[\hspace*{54pt}] {\it statement} is a statement, which should be
executed under control of the for statement;
\item[\hspace*{54pt}] {\it statement}, \dots, {\it statement} is a
sequence of statements (enclosed in curly braces), which should be
executed under control of the for statement.
\end{description}

\begin{description}
\item[{\rm Note:}\hspace*{31pt}] Only the following statements can be
used within the for statement: check, display, printf, and another for.
\end{description}

\noindent{\bf Examples}

\begin{verbatim}
for {(i,j) in E: i != j}
{  printf "flow from %s to %s is %g\n", i, j, x[i,j];
   check x[i,j] >= 0;
}
\end{verbatim}

\newpage

\begin{verbatim}
for {i in 1..n}
{  for {j in 1..n} printf " %s", if x[i,j] then "Q" else ".";
   printf("\n");
}
for {1..72} printf("*");
\end{verbatim}

The for statement causes a statement or a sequence of statements
specified as part of the for statement to be executed for every
$n$-tuple in the domain set. Thus, statements within the for statement
may include dummy indices introduced in corresponding indexing
expression.

\subsection{Table statement}

\medskip

\framebox[345pt][l]{
\parbox[c][68pt]{345pt}{
\hspace{6pt} {\tt table} {\it name} {\it alias} {\tt IN} {\it driver}
{\it arg} \dots {\it arg} {\tt:}

\hspace{6pt} {\tt\ \ \ \ \ } {\it set} {\tt<-} {\tt[} {\it fld} {\tt,}
\dots {\tt,} {\it fld} {\tt]} {\tt,} {\it par} {\tt\textasciitilde}
{\it fld} {\tt,} \dots {\tt,} {\it par} {\tt\textasciitilde} {\it fld}
{\tt;}

\medskip

\hspace{6pt} {\tt table} {\it name} {\it alias} {\it domain} {\tt OUT}
{\it driver} {\it arg} \dots {\it arg} {\tt:}

\hspace{6pt} {\tt\ \ \ \ \ } {\it expr} {\tt\textasciitilde} {\it fld}
{\tt,} \dots {\tt,} {\it expr} {\tt\textasciitilde} {\it fld} {\tt;}
}}

\setlength{\leftmargini}{60pt}

\begin{description}
\item[{\rm Where:}\hspace*{23pt}] {\it name} is a symbolic name of the
table;
\item[\hspace*{54pt}] {\it alias} is an optional string literal, which
specifies an alias of the table;
\item[\hspace*{54pt}] {\it domain} is an indexing expression, which
specifies a subscript domain of the (output) table;
\item[\hspace*{54pt}] {\tt IN} means reading data from the input table;
\item[\hspace*{54pt}] {\tt OUT} means writing data to the output table;
\item[\hspace*{54pt}] {\it driver} is a symbolic expression, which
specifies the driver used to access the table (for details see Section
\ref{drivers}, page \pageref{drivers});
\item[\hspace*{54pt}] {\it arg} is an optional symbolic expression,
which is an argument pass\-ed to the table driver. This symbolic
expression must not include dummy indices specified in the domain;
\item[\hspace*{54pt}] {\it set} is the name of an optional simple set
called {\it control set}. It can be omitted along with the delimiter
{\tt<-};
\item[\hspace*{54pt}] {\it fld} is a field name. Within square brackets
at least one field should be specified. The field name following
a parameter name or expression is optional and can be omitted along
with the delimiter {\tt\textasciitilde}, in which case the name of
corresponding model object is used as the field name;
\item[\hspace*{54pt}] {\it par} is a symbolic name of a model parameter;
\item[\hspace*{54pt}] {\it expr} is a numeric or symbolic expression.
\end{description}

\newpage

\noindent{\bf Examples}

\begin{verbatim}
table data IN "CSV" "data.csv":
   S <- [FROM,TO], d~DISTANCE, c~COST;
table result{(f,t) in S} OUT "CSV" "result.csv":
   f~FROM, t~TO, x[f,t]~FLOW;
\end{verbatim}

The table statement allows reading data from a table into model
objects such as sets and (non-scalar) parameters as well as writing
data from the model to a table.

\subsubsection{Table structure}

A {\it data table} is an (unordered) set of {\it records}, where each
record consists of the same number of {\it fields}, and each field is
provided with a unique symbolic name called the {\it field name}. For
example:

\bigskip

\begin{tabular}{@{\hspace*{38mm}}c@{\hspace*{11mm}}c@{\hspace*{10mm}}c
@{\hspace*{9mm}}c}
First&Second&&Last\\
field&field&.\ \ .\ \ .&field\\
$\downarrow$&$\downarrow$&&$\downarrow$\\
\end{tabular}

\begin{tabular}{ll@{}}
Table header&$\rightarrow$\\
First record&$\rightarrow$\\
Second record&$\rightarrow$\\
\\
\hfil .\ \ .\ \ .\\
\\
Last record&$\rightarrow$\\
\end{tabular}
\begin{tabular}{|l|l|c|c|}
\hline
{\tt FROM}&{\tt TO}&{\tt DISTANCE}&{\tt COST}\\
\hline
{\tt Seattle}  &{\tt New-York}&{\tt 2.5}&{\tt 0.12}\\
{\tt Seattle}  &{\tt Chicago} &{\tt 1.7}&{\tt 0.08}\\
{\tt Seattle}  &{\tt Topeka}  &{\tt 1.8}&{\tt 0.09}\\
{\tt San-Diego}&{\tt New-York}&{\tt 2.5}&{\tt 0.15}\\
{\tt San-Diego}&{\tt Chicago} &{\tt 1.8}&{\tt 0.10}\\
{\tt San-Diego}&{\tt Topeka}  &{\tt 1.4}&{\tt 0.07}\\
\hline
\end{tabular}

\subsubsection{Reading data from input table}

The input table statement causes reading data from the specified table
record by record.

Once a next record has been read, numeric or symbolic values of fields,
whose names are enclosed in square brackets in the table statement, are
gathered into $n$-tuple, and if the control set is specified in the
table statement, this $n$-tuple is added to it. Besides, a numeric or
symbolic value of each field associated with a model parameter is
assigned to the parameter member identified by subscripts, which are
components of the $n$-tuple just read.

For example, the following input table statement:

\medskip

\noindent\hfil
\verb|table data IN "...": S <- [FROM,TO], d~DISTANCE, c~COST;|

\medskip

\noindent
causes reading values of four fields named {\tt FROM}, {\tt TO},
{\tt DISTANCE}, and {\tt COST} from each record of the specified table.
Values of fields {\tt FROM} and {\tt TO} give a pair $(f,t)$, which is
added to the control set {\tt S}. The value of field {\tt DISTANCE} is
assigned to parameter member ${\tt d}[f,t]$, and the value of field
{\tt COST} is assigned to parameter member ${\tt c}[f,t]$.

Note that the input table may contain extra fields whose names are not
specified in the table statement, in which case values of these fields
on reading the table are ignored.

\subsubsection{Writing data to output table}

The output table statement causes writing data to the specified table.
Note that some drivers (namely, CSV and xBASE) destroy the output table
before writing data, i.e. delete all its existing records.

Each $n$-tuple in the specified domain set generates one record written
to the output table. Values of fields are numeric or symbolic values of
corresponding expressions specified in the table statement. These
expressions are evaluated for each $n$-tuple in the domain set and,
thus, may include dummy indices introduced in the corresponding indexing
expression.

For example, the following output table statement:

\medskip

\noindent
\verb|   table result{(f,t) in S} OUT "...": f~FROM, t~TO, x[f,t]~FLOW;|

\medskip

\noindent
causes writing records, by one record for each pair $(f,t)$ in set
{\tt S}, to the output table, where each record consists of three
fields named {\tt FROM}, {\tt TO}, and {\tt FLOW}. The values written
to fields {\tt FROM} and {\tt TO} are current values of dummy indices
{\tt f} and {\tt t}, and the value written to field {\tt FLOW} is
a value of member ${\tt x}[f,t]$ of corresponding subscripted parameter
or variable.

%%%%%%%%%%%%%%%%%%%%%%%%%%%%%%%%%%%%%%%%%%%%%%%%%%%%%%%%%%%%%%%%%%%%%%%%

\newpage

\section{Model data}

{\it Model data} include elemental sets, which are ``values'' of model
sets, and numeric and symbolic values of model parameters.

In MathProg there are two different ways to saturate model sets and
parameters with data. One way is simply providing necessary data using
the assign attribute. However, in many cases it is more practical to
separate the model itself and particular data needed for the model. For
the latter reason in MathProg there is another way, when the model
description is divided into two parts: model section and data section.

A {\it model section} is a main part of the model description that
contains declarations of all model objects and is common for all
problems based on that model.

A {\it data section} is an optional part of the model description that
contains model data specific for a particular problem.

In MathProg model and data sections can be placed either in one text
file or in two separate text files.

1. If both model and data sections are placed in one file, the file is
composed as follows:

\bigskip

\noindent\hfil
\framebox{\begin{tabular}{l}
{\it statement}{\tt;}\\
{\it statement}{\tt;}\\
\hfil.\ \ .\ \ .\\
{\it statement}{\tt;}\\
{\tt data;}\\
{\it data block}{\tt;}\\
{\it data block}{\tt;}\\
\hfil.\ \ .\ \ .\\
{\it data block}{\tt;}\\
{\tt end;}
\end{tabular}}

\bigskip

2. If the model and data sections are placed in two separate files, the
files are composed as follows:

\bigskip

\noindent\hfil
\begin{tabular}{@{}c@{}}
\framebox{\begin{tabular}{l}
{\it statement}{\tt;}\\
{\it statement}{\tt;}\\
\hfil.\ \ .\ \ .\\
{\it statement}{\tt;}\\
{\tt end;}\\
\end{tabular}}\\
\\\\Model file\\
\end{tabular}
\hspace{32pt}
\begin{tabular}{@{}c@{}}
\framebox{\begin{tabular}{l}
{\tt data;}\\
{\it data block}{\tt;}\\
{\it data block}{\tt;}\\
\hfil.\ \ .\ \ .\\
{\it data block}{\tt;}\\
{\tt end;}\\
\end{tabular}}\\
\\Data file\\
\end{tabular}

\bigskip

\begin{description}
\item[{\rm Note:}\hspace*{31pt}] If the data section is placed in a
separate file, the keyword {\tt data} is optional and may be omitted
along with the semicolon that follows it.
\end{description}

\subsection{Coding data section}

The {\it data section} is a sequence of data blocks in various formats,
which are discussed in following subsections. The order, in which data
blocks follow in the data section, may be arbitrary, not necessarily
the same, in which corresponding model objects follow in the model
section.

The rules of coding the data section are commonly the same as the rules
of coding the model description (see Subsection \ref{coding}, page
\pageref{coding}), i.e. data blocks are composed from basic lexical
units such as symbolic names, numeric and string literals, keywords,
delimiters, and comments. However, for the sake of convenience and
improving readability there is one deviation from the common rule: if
a string literal consists of only alphanumeric characters (including
the underscore character), the signs {\tt+} and {\tt-}, and/or the
decimal point, it may be coded without bordering by (single or double)
quotes.

All numeric and symbolic material provided in the data section is coded
in the form of numbers and symbols, i.e. unlike the model section
no expressions are allowed in the data section. Nevertheless, the signs
{\tt+} and {\tt-} can precede numeric literals to allow coding signed
numeric quantities, in which case there must be no white-space
characters between the sign and following numeric literal (if there is
at least one white-space, the sign and following numeric literal are
recognized as two different lexical units).

\subsection{Set data block}

\medskip

\framebox[345pt][l]{
\parbox[c][44pt]{345pt}{
\hspace{6pt} {\tt set} {\it name} {\tt,} {\it record} {\tt,} \dots
{\tt,} {\it record} {\tt;}

\medskip

\hspace{6pt} {\tt set} {\it name} {\tt[} {\it symbol} {\tt,} \dots
{\tt,} {\it symbol} {\tt]} {\tt,} {\it record} {\tt,} \dots {\tt,}
{\it record} {\tt;}
}}

\setlength{\leftmargini}{60pt}

\begin{description}
\item[{\rm Where:}\hspace*{23pt}] {\it name} is a symbolic name of the
set;
\item[\hspace*{54pt}] {\it symbol}, \dots, {\it symbol} are subscripts,
which specify a particular member of the set (if the set is an array,
i.e. a set of sets);
\item[\hspace*{54pt}] {\it record}, \dots, {\it record} are data
records.
\end{description}

\begin{description}
\item[{\rm Note:}\hspace*{31pt}] Commae preceding data records may be
omitted.
\end{description}

\noindent Data records:

\begin{description}
\item[{\tt :=}\hspace*{45pt}] is a non-significant data record, which
may be used freely to improve readability;
\item[{\tt(} {\it slice} {\tt)}\hspace*{18.5pt}] specifies a slice;
\item[{\it simple-data}\hspace*{5.5pt}] specifies set data in the
simple format;
\item[{\tt:} {\it matrix-data}]\hspace*{0pt}\\
specifies set data in the matrix format;
\item[{\tt(tr)} {\tt:} {\it matrix-data}]\hspace*{0pt}\\
specifies set data in the transposed matrix format. (In this case the
colon following the keyword {\tt(tr)} may be omitted.)
\end{description}

\noindent{\bf Examples}

\begin{verbatim}
set month := Jan Feb Mar Apr May Jun;
set month "Jan", "Feb", "Mar", "Apr", "May", "Jun";
set A[3,Mar] := (1,2) (2,3) (4,2) (3,1) (2,2) (4,4) (3,4);
set A[3,'Mar'] := 1 2 2 3 4 2 3 1 2 2 4 4 2 4;
set A[3,'Mar'] : 1 2 3 4 :=
               1 - + - -
               2 - + + -
               3 + - - +
               4 - + - + ;
set B := (1,2,3) (1,3,2) (2,3,1) (2,1,3) (1,2,2) (1,1,1) (2,1,1);
set B := (*,*,*) 1 2 3, 1 3 2, 2 3 1, 2 1 3, 1 2 2, 1 1 1, 2 1 1;
set B := (1,*,2) 3 2 (2,*,1) 3 1 (1,2,3) (2,1,3) (1,1,1);
set B := (1,*,*) : 1 2 3 :=
                 1 + - -
                 2 - + +
                 3 - + -
         (2,*,*) : 1 2 3 :=
                 1 + - +
                 2 - - -
                 3 + - - ;
\end{verbatim}

\noindent(In these examples {\tt month} is a simple set of singlets,
{\tt A} is a 2-dimensional array of doublets, and {\tt B} is a simple
set of triplets. Data blocks for the same set are equivalent in the
sense that they specify the same data in different formats.)

\medskip

The {\it set data block} is used to specify a complete elemental set,
which is assigned to a set (if it is a simple set) or one of its
members (if the set is an array of sets).\footnote{There is another way
to specify data for a simple set along with data for parameters. This
feature is discussed in the next subsection.}

Data blocks can be specified only for non-computable sets, i.e. for
sets, which have no assign ({\tt:=}) attribute in the corresponding set
statements.

If the set is a simple set, only its symbolic name should be specified
in the header of the data block. Otherwise, if the set is a
$n$-dimensional array, its symbolic name should be provided with a
complete list of subscripts separated by commae and enclosed in square
brackets to specify a particular member of the set array. The number of
subscripts must be the same as the dimension of the set array, where
each subscript must be a number or symbol.

An elemental set defined in the set data block is coded as a sequence
of data records described below.\footnote{{\it Data record} is simply a
technical term. It does not mean that data records have any special
formatting.}

\newpage

\subsubsection{Assign data record}

The {\it assign} ({\tt:=}) {\it data record} is a non-signficant
element. It may be used for improving readability of data blocks.

\subsubsection{Slice data record}

The {\it slice data record} is a control record, which specifies a
{\it slice} of the elemental set defined in the data block. It has the
following syntactic form:

\medskip

\noindent\hfil
{\tt(} $s_1$ {\tt,} $s_2$ {\tt,} \dots {\tt,} $s_n$ {\tt)}

\medskip

\noindent where $s_1$, $s_2$, \dots, $s_n$ are components of the slice.

Each component of the slice can be a number or symbol or the asterisk
({\tt*}). The number of components in the slice must be the same as the
dimension of $n$-tuples in the elemental set to be defined. For
instance, if the elemental set contains 4-tuples (quadruplets), the
slice must have four components. The number of asterisks in the slice
is called the {\it slice dimension}.

The effect of using slices is the following. If a $m$-dimensional slice
(i.e. a slice having $m$ asterisks) is specified in the data block, all
subsequent data records must specify tuples of the dimension $m$.
Whenever a $m$-tuple is encountered, each asterisk in the slice is
replaced by corresponding components of the $m$-tuple that gives the
resultant $n$-tuple, which is included in the elemental set to be
defined. For example, if the slice $(a,*,1,2,*)$ is in effect, and
2-tuple $(3,b)$ is encountered in a subsequent data record, the
resultant 5-tuple included in the elemental set is $(a,3,1,2,b)$.

The slice having no asterisks itself defines a complete $n$-tuple,
which is included in the elemental set.

Being once specified the slice effects until either a new slice or the
end of data block is encountered. Note that if no slice is specified in
the data block, one, components of which are all asterisks, is assumed.

\subsubsection{Simple data record}

The {\it simple data record} defines one $n$-tuple in a simple format
and has the following syntactic form:

\medskip

\noindent\hfil
$t_1$ {\tt,} $t_2$ {\tt,} \dots {\tt,} $t_n$

\medskip

\noindent where $t_1$, $t_2$, \dots, $t_n$ are components of the
$n$-tuple. Each component can be a number or symbol. Commae between
components are optional and may be omitted.

\subsubsection{Matrix data record}

The {\it matrix data record} defines several 2-tuples (doublets) in
a matrix format and has the following syntactic form:

\newpage

$$\begin{array}{cccccc}
\mbox{{\tt:}}&c_1&c_2&\dots&c_n&\mbox{{\tt:=}}\\
r_1&a_{11}&a_{12}&\dots&a_{1n}&\\
r_2&a_{21}&a_{22}&\dots&a_{2n}&\\
\multicolumn{5}{c}{.\ \ .\ \ .\ \ .\ \ .\ \ .\ \ .\ \ .\ \ .}&\\
r_m&a_{m1}&a_{m2}&\dots&a_{mn}&\\
\end{array}$$
where $r_1$, $r_2$, \dots, $r_m$ are numbers and/or symbols
corresponding to rows of the matrix; $c_1$, $c_2$, \dots, $c_n$ are
numbers and/or symbols corresponding to columns of the matrix, $a_{11}$,
$a_{12}$, \dots, $a_{mn}$ are matrix elements, which can be either
{\tt+} or {\tt-}. (In this data record the delimiter {\tt:} preceding
the column list and the delimiter {\tt:=} following the column list
cannot be omitted.)

Each element $a_{ij}$ of the matrix data block (where $1\leq i\leq m$,
$1\leq j\leq n$) corresponds to 2-tuple $(r_i,c_j)$. If $a_{ij}$ is the
plus sign ({\tt+}), that 2-tuple (or a longer $n$-tuple, if a slice is
used) is included in the elemental set. Otherwise, if $a_{ij}$ is the
minus sign ({\tt-}), that 2-tuple is not included in the elemental set.

Since the matrix data record defines 2-tuples, either the elemental set
must consist of 2-tuples or the slice currently used must be
2-dimensional.

\subsubsection{Transposed matrix data record}

The {\it transposed matrix data record} has the following syntactic
form:
$$\begin{array}{cccccc}
\mbox{{\tt(tr) :}}&c_1&c_2&\dots&c_n&\mbox{{\tt:=}}\\
r_1&a_{11}&a_{12}&\dots&a_{1n}&\\
r_2&a_{21}&a_{22}&\dots&a_{2n}&\\
\multicolumn{5}{c}{.\ \ .\ \ .\ \ .\ \ .\ \ .\ \ .\ \ .\ \ .}&\\
r_m&a_{m1}&a_{m2}&\dots&a_{mn}&\\
\end{array}$$
(In this case the delimiter {\tt:} following the keyword {\tt(tr)} is
optional and may be omitted.)

This data record is completely analogous to the matrix data record (see
above) with only exception that in this case each element $a_{ij}$ of
the matrix corresponds to 2-tuple $(c_j,r_i)$ rather than $(r_i,c_j)$.

Being once specified the {\tt(tr)} indicator affects all subsequent
data records until either a slice or the end of data block is
encountered.

\subsection{Parameter data block}

\medskip

\framebox[345pt][l]{
\parbox[c][80pt]{345pt}{
\hspace{6pt} {\tt param} {\it name} {\tt,} {\it record} {\tt,} \dots
{\tt,} {\it record} {\tt;}

\medskip

\hspace{6pt} {\tt param} {\it name} {\tt default} {\it value} {\tt,}
{\it record} {\tt,} \dots {\tt,} {\it record} {\tt;}

\medskip

\hspace{6pt} {\tt param} {\tt:} {\it tabbing-data} {\tt;}

\medskip

\hspace{6pt} {\tt param} {\tt default} {\it value} {\tt:}
{\it tabbing-data} {\tt;}
}}

\newpage

\setlength{\leftmargini}{60pt}

\begin{description}
\item[{\rm Where:}\hspace*{23pt}] {\it name} is a symbolic name of the
parameter;
\item[\hspace*{54pt}] {\it value} is an optional default value of the
parameter;
\item[\hspace*{54pt}] {\it record}, \dots, {\it record} are data
records;
\item[\hspace*{54pt}] {\it tabbing-data} specifies parameter data in
the tabbing format.
\end{description}

\begin{description}
\item[{\rm Note:}\hspace*{31pt}] Commae preceding data records may be
omitted.
\end{description}

\noindent Data records:

\begin{description}
\item[{\tt :=}\hspace*{45pt}] is a non-significant data record, which
may be used freely to improve readability;
\item[{\tt[} {\it slice} {\tt]}\hspace*{18.5pt}] specifies a slice;
\item[{\it plain-data}\hspace*{11pt}] specifies parameter data in the
plain format;
\item[{\tt:} {\it tabular-data}]\hspace*{0pt}\\
specifies parameter data in the tabular format;
\item[{\tt(tr)} {\tt:} {\it tabular-data}]\hspace*{0pt}\\
specifies set data in the transposed tabular format. (In this case the
colon following the keyword {\tt(tr)} may be omitted.)
\end{description}

\noindent{\bf Examples}

\begin{verbatim}
param T := 4;
param month := 1 'Jan' 2 'Feb' 3 'Mar' 4 'Apr' 5 'May';
param month := [1] Jan, [2] Feb, [3] Mar, [4] Apr, [5] May;
param day := [Sun] 0, [Mon] 1, [Tue] 2, [Wed] 3, [Thu] 4,
             [Fri] 5, [Sat] 6;
param init_stock := iron 7.32 nickel 35.8;
param init_stock [*] iron 7.32, nickel 35.8;
param cost [iron] .025 [nickel] .03;
param value := iron -.1, nickel .02;
param       : init_stock cost value :=
      iron       7.32    .025  -.1
      nickel    35.8     .03    .02 ;
param : raw : init_stock cost value :=
      iron       7.32    .025  -.1
      nickel    35.8     .03    .02 ;
param demand default 0 (tr)
       :  FRA  DET  LAN  WIN  STL  FRE  LAF :=
   bands  300   .   100   75   .   225  250
   coils  500  750  400  250   .   850  500
   plate  100   .    .    50  200   .   250 ;
\end{verbatim}

\newpage

\begin{verbatim}
param trans_cost :=
   [*,*,bands]:  FRA  DET  LAN  WIN  STL  FRE  LAF :=
         GARY     30   10    8   10   11   71    6
         CLEV     22    7   10    7   21   82   13
         PITT     19   11   12   10   25   83   15
   [*,*,coils]:  FRA  DET  LAN  WIN  STL  FRE  LAF :=
         GARY     39   14   11   14   16   82    8
         CLEV     27    9   12    9   26   95   17
         PITT     24   14   17   13   28   99   20
   [*,*,plate]:  FRA  DET  LAN  WIN  STL  FRE  LAF :=
         GARY     41   15   12   16   17   86    8
         CLEV     29    9   13    9   28   99   18
         PITT     26   14   17   13   31  104   20 ;
\end{verbatim}

The {\it parameter data block} is used to specify complete data for a
parameter (or parameters, if data are specified in the tabbing format).

Data blocks can be specified only for non-computable parameters, i.e.
for parameters, which have no assign ({\tt:=}) attribute in the
corresponding parameter statements.

Data defined in the parameter data block are coded as a sequence of
data records described below. Additionally the data block can be
provided with the optional {\tt default} attribute, which specifies a
default numeric or symbolic value of the parameter (parameters). This
default value is assigned to the parameter or its members, if
no appropriate value is defined in the parameter data block. The
{\tt default} attribute cannot be used, if it is already specified in
the corresponding parameter statement.

\subsubsection{Assign data record}

The {\it assign} ({\tt:=}) {\it data record} is a non-signficant
element. It may be used for improving readability of data blocks.

\subsubsection{Slice data record}

The {\it slice data record} is a control record, which specifies a
{\it slice} of the parameter array. It has the following syntactic form:

\medskip

\noindent\hfil
{\tt[} $s_1$ {\tt,} $s_2$ {\tt,} \dots {\tt,} $s_n$ {\tt]}

\medskip

\noindent where $s_1$, $s_2$, \dots, $s_n$ are components of the slice.

Each component of the slice can be a number or symbol or the asterisk
({\tt*}). The number of components in the slice must be the same as the
dimension of the parameter. For instance, if the parameter is a
4-dimensional array, the slice must have four components. The number of
asterisks in the slice is called the {\it slice dimension}.

The effect of using slices is the following. If a $m$-dimensional slice
(i.e. a slice having $m$ asterisks) is specified in the data block, all
subsequent data records must specify subscripts of the parameter
members as if the parameter were $m$-dimensional, not $n$-dimensional.

Whenever $m$ subscripts are encountered, each asterisk in the slice is
replaced by corresponding subscript that gives $n$ subscripts, which
define the actual parameter member. For example, if the slice
$[a,*,1,2,*]$ is in effect, and subscripts 3 and $b$ are encountered in
a subsequent data record, the complete subscript list used to choose a
parameter member is $[a,3,1,2,b]$.

It is allowed to specify a slice having no asterisks. Such slice itself
defines a complete subscript list, in which case the next data record
should define only a single value of corresponding parameter member.

Being once specified the slice effects until either a new slice or the
end of data block is encountered. Note that if no slice is specified in
the data block, one, components of which are all asterisks, is assumed.

\subsubsection{Plain data record}

The {\it plain data record} defines a subscript list and a single value
in the plain format. This record has the following syntactic form:

\medskip

\noindent\hfil
$t_1$ {\tt,} $t_2$ {\tt,} \dots {\tt,} $t_n$ {\tt,} $v$

\medskip

\noindent where $t_1$, $t_2$, \dots, $t_n$ are subscripts, and $v$ is a
value. Each subscript as well as the value can be a number or symbol.
Commae following subscripts are optional and may be omitted.

In case of 0-dimensional parameter or slice the plain data record has
no subscripts and consists of a single value only.

\subsubsection{Tabular data record}

The {\it tabular data record} defines several values, where each value
is provided with two subscripts. This record has the following
syntactic form:
$$\begin{array}{cccccc}
\mbox{{\tt:}}&c_1&c_2&\dots&c_n&\mbox{{\tt:=}}\\
r_1&a_{11}&a_{12}&\dots&a_{1n}&\\
r_2&a_{21}&a_{22}&\dots&a_{2n}&\\
\multicolumn{5}{c}{.\ \ .\ \ .\ \ .\ \ .\ \ .\ \ .\ \ .\ \ .}&\\
r_m&a_{m1}&a_{m2}&\dots&a_{mn}&\\
\end{array}$$
where $r_1$, $r_2$, \dots, $r_m$ are numbers and/or symbols
corresponding to rows of the table; $c_1$, $c_2$, \dots, $c_n$ are
numbers and/or symbols corresponding to columns of the table, $a_{11}$,
$a_{12}$, \dots, $a_{mn}$ are table elements. Each element can be a
number or symbol or the single decimal point ({\tt.}). (In this data
record the delimiter {\tt:} preceding the column list and the delimiter
{\tt:=} following the column list cannot be omitted.)

Each element $a_{ij}$ of the tabular data block ($1\leq i\leq m$,
$1\leq j\leq n$) defines two subscripts, where the first subscript is
$r_i$, and the second one is $c_j$. These subscripts are used in
conjunction with the current slice to form the complete subscript list
that identifies a particular member of the parameter array. If $a_{ij}$
is a number or symbol, this value is assigned to the parameter member.
However, if $a_{ij}$ is the single decimal point, the member is
assigned a default value specified either in the parameter data block
or in the parameter statement, or, if no default value is specified,
the member remains undefined.

Since the tabular data record provides two subscripts for each value,
either the parameter or the slice currently used must be 2-dimensional.

\subsubsection{Transposed tabular data record}

The {\it transposed tabular data record} has the following syntactic
form:
$$\begin{array}{cccccc}
\mbox{{\tt(tr) :}}&c_1&c_2&\dots&c_n&\mbox{{\tt:=}}\\
r_1&a_{11}&a_{12}&\dots&a_{1n}&\\
r_2&a_{21}&a_{22}&\dots&a_{2n}&\\
\multicolumn{5}{c}{.\ \ .\ \ .\ \ .\ \ .\ \ .\ \ .\ \ .\ \ .}&\\
r_m&a_{m1}&a_{m2}&\dots&a_{mn}&\\
\end{array}$$
(In this case the delimiter {\tt:} following the keyword {\tt(tr)} is
optional and may be omitted.)

This data record is completely analogous to the tabular data record
(see above) with only exception that the first subscript defined by
element $a_{ij}$ is $c_j$ while the second one is $r_i$.

Being once specified the {\tt(tr)} indicator affects all subsequent
data records until either a slice or the end of data block is
encountered.

\subsubsection{Tabbing data format}

The parameter data block in the {\it tabbing format} has the following
syntactic form:
$$\begin{array}{p{12pt}@{\ }l@{\ }c@{\ }l@{\ }c@{\ }l@{\ }r@{\ }l@{\ }c
@{\ }l@{\ }c@{\ }l@{\ }l}
\multicolumn{7}{@{}c@{}}{\mbox{\tt param}\ \mbox{\tt default}\ \mbox
{\it value}\ \mbox{\tt:}\ \mbox{\it s}\ \mbox{\tt:}}&
p_1&\mbox{\tt,}&p_2&\mbox{\tt,} \dots \mbox{\tt,}&p_k&\mbox{\tt:=}\\
&t_{11}&\mbox{\tt,}&t_{12}&\mbox{\tt,} \dots \mbox{\tt,}&t_{1n}&
\mbox{\tt,}&a_{11}&\mbox{\tt,}&a_{12}&\mbox{\tt,} \dots \mbox{\tt,}&
a_{1k}\\
&t_{21}&\mbox{\tt,}&t_{22}&\mbox{\tt,} \dots \mbox{\tt,}&t_{2n}&
\mbox{\tt,}&a_{21}&\mbox{\tt,}&a_{22}&\mbox{\tt,} \dots \mbox{\tt,}&
a_{2k}\\
\multicolumn{13}{c}
{.\ \ .\ \ .\ \ .\ \ .\ \ .\ \ .\ \ .\ \ .\ \ .\ \ .\ \ .\ \ .\ \ .}\\
&t_{m1}&\mbox{\tt,}&t_{m2}&\mbox{\tt,} \dots \mbox{\tt,}&t_{mn}&
\mbox{\tt,}&a_{m1}&\mbox{\tt,}&a_{m2}&\mbox{\tt,} \dots \mbox{\tt,}&
a_{mk}&\mbox{\tt;}\\
\end{array}$$

{\it Notes:}

1. The keyword {\tt default} may be omitted along with a value
following it.

2. Symbolic name {\tt s} may be omitted along with the colon following
it.

3. All comae are optional and may be omitted.

\medskip

The data block in the tabbing format shown above is exactly equivalent
to the following data blocks for $j=1,2,\dots,k$:

\medskip

{\tt set} {\it s} {\tt:=}
{\tt(}$t_{11}${\tt,}$t_{12}${\tt,}\dots{\tt,}$t_{1n}${\tt)}
{\tt(}$t_{21}${\tt,}$t_{22}${\tt,}\dots{\tt,}$t_{2n}${\tt)} \dots
{\tt(}$t_{m1}${\tt,}$t_{m2}${\tt,}\dots{\tt,}$t_{mn}${\tt)} {\tt;}

{\tt param} $p_j$ {\tt default} {\it value} {\tt:=}

$\!${\tt[}$t_{11}${\tt,}$t_{12}${\tt,}\dots{\tt,}$t_{1n}${\tt]}
$a_{1j}$
{\tt[}$t_{21}${\tt,}$t_{22}${\tt,}\dots{\tt,}$t_{2n}${\tt]} $a_{2j}$
\dots
{\tt[}$t_{m1}${\tt,}$t_{m2}${\tt,}\dots{\tt,}$t_{mn}${\tt]} $a_{mj}$
{\tt;}

%%%%%%%%%%%%%%%%%%%%%%%%%%%%%%%%%%%%%%%%%%%%%%%%%%%%%%%%%%%%%%%%%%%%%%%%

\appendix

\newpage

\section{Using suffixes}

Suffixes can be used to retrieve additional values associated with
model variables, constraints, and objectives.

A {\it suffix} consists of a period ({\tt.}) followed by a non-reserved
keyword. For example, if {\tt x} is a two-dimensional variable,
{\tt x[i,j].lb} is a numeric value equal to the lower bound of
elemental variable {\tt x[i,j]}, which (value) can be used everywhere
in expressions like a numeric parameter.

For model variables suffixes have the following meaning:

\medskip

\begin{tabular}{@{}p{96pt}p{222pt}@{}}
{\tt.lb}&lower bound\\
{\tt.ub}&upper bound\\
{\tt.status}&status in the solution:\\
&0 --- undefined\\
&1 --- basic\\
&2 --- non-basic on lower bound\\
&3 --- non-basic on upper bound\\
&4 --- non-basic free (unbounded) variable\\
&5 --- non-basic fixed variable\\
{\tt.val}&primal value in the solution\\
{\tt.dual}&dual value (reduced cost) in the solution\\
\end{tabular}

\medskip

For model constraints and objectives suffixes have the following
meaning:

\medskip

\begin{tabular}{@{}p{96pt}p{222pt}@{}}
{\tt.lb}&lower bound of the linear form\\
{\tt.ub}&upper bound of the linear form\\
{\tt.status}&status in the solution:\\
&0 --- undefined\\
&1 --- non-active\\
&2 --- active on lower bound\\
&3 --- active on upper bound\\
&4 --- active free (unbounded) row\\
&5 --- active equality constraint\\
{\tt.val}&primal value of the linear form in the solution\\
{\tt.dual}&dual value (reduced cost) of the linear form in the
solution\\
\end{tabular}

\medskip

Note that suffixes {\tt.status}, {\tt.val}, and {\tt.dual} can be used
only below the solve statement.

%%%%%%%%%%%%%%%%%%%%%%%%%%%%%%%%%%%%%%%%%%%%%%%%%%%%%%%%%%%%%%%%%%%%%%%%

\newpage

\section{Date and time functions}

\noindent\hfil
by Andrew Makhorin \verb|<mao@gnu.org>|

\noindent\hfil
and Heinrich Schuchardt \verb|<heinrich.schuchardt@gmx.de>|

\subsection{Obtaining current calendar time}
\label{gmtime}

To obtain the current calendar time in MathProg there exists the
function {\tt gmtime}. It has no arguments and returns the number of
seconds elapsed since 00:00:00 on January 1, 1970, Coordinated
Universal Time (UTC). For example:

\medskip

\verb|   param utc := gmtime();|

\medskip

MathProg has no function to convert UTC time returned by the function
{\tt gmtime} to {\it local} calendar times. Thus, if you need to
determine the current local calendar time, you have to add to the UTC
time returned the time offset from UTC expressed in seconds. For
example, the time in Berlin during the winter is one hour ahead of UTC
that corresponds to the time offset +1 hour = +3600 secs, so the
current winter calendar time in Berlin may be determined as follows:

\medskip

\verb|   param now := gmtime() + 3600;|

\medskip

\noindent Similarly, the summer time in Chicago (Central Daylight Time)
is five hours behind UTC, so the corresponding current local calendar
time may be determined as follows:

\medskip

\verb|   param now := gmtime() - 5 * 3600;|

\medskip

Note that the value returned by {\tt gmtime} is volatile, i.e. being
called several times this function may return different values.

\subsection{Converting character string to calendar time}
\label{str2time}

The function {\tt str2time(}{\it s}{\tt,} {\it f}{\tt)} converts a
character string (timestamp) specified by its first argument {\it s},
which must be a symbolic expression, to the calendar time suitable for
arithmetic calculations. The conversion is controlled by the specified
format string {\it f} (the second argument), which also must be a
symbolic expression.

The result of conversion returned by {\tt str2time} has the same
meaning as values returned by the function {\tt gmtime} (see Subsection
\ref{gmtime}, page \pageref{gmtime}). Note that {\tt str2time} does
{\tt not} correct the calendar time returned for the local timezone,
i.e. being applied to 00:00:00 on January 1, 1970 it always returns 0.

For example, the model statements:

\medskip

\verb|   param s, symbolic, := "07/14/98 13:47";|

\verb|   param t := str2time(s, "%m/%d/%y %H:%M");|

\verb|   display t;|

\medskip

\noindent produce the following printout:

\medskip

\verb|   t = 900424020|

\medskip

\noindent where the calendar time printed corresponds to 13:47:00 on
July 14, 1998.

\newpage

The format string passed to the function {\tt str2time} consists of
conversion specifiers and ordinary characters. Each conversion
specifier begins with a percent ({\tt\%}) character followed by a
letter.

The following conversion specifiers may be used in the format string:

\medskip

\begin{tabular}{@{}p{20pt}p{298pt}@{}}
{\tt\%b}&The abbreviated month name (case insensitive). At least three
first letters of the month name must appear in the input string.\\
\end{tabular}

\medskip

\begin{tabular}{@{}p{20pt}p{298pt}@{}}
{\tt\%d}&The day of the month as a decimal number (range 1 to 31).
Leading zero is permitted, but not required.\\
\end{tabular}

\medskip

\begin{tabular}{@{}p{20pt}p{298pt}@{}}
{\tt\%h}&The same as {\tt\%b}.\\
\end{tabular}

\medskip

\begin{tabular}{@{}p{20pt}p{298pt}@{}}
{\tt\%H}&The hour as a decimal number, using a 24-hour clock (range 0
to 23). Leading zero is permitted, but not required.\\
\end{tabular}

\medskip

\begin{tabular}{@{}p{20pt}p{298pt}@{}}
{\tt\%m}&The month as a decimal number (range 1 to 12). Leading zero is
permitted, but not required.\\
\end{tabular}

\medskip

\begin{tabular}{@{}p{20pt}p{298pt}@{}}
{\tt\%M}&The minute as a decimal number (range 0 to 59). Leading zero
is permitted, but not required.\\
\end{tabular}

\medskip

\begin{tabular}{@{}p{20pt}p{298pt}@{}}
{\tt\%S}&The second as a decimal number (range 0 to 60). Leading zero
is permitted, but not required.\\
\end{tabular}

\medskip

\begin{tabular}{@{}p{20pt}p{298pt}@{}}
{\tt\%y}&The year without a century as a decimal number (range 0 to 99).
Leading zero is permitted, but not required. Input values in the range
0 to 68 are considered as the years 2000 to 2068 while the values 69 to
99 as the years 1969 to 1999.\\
\end{tabular}

\medskip

\begin{tabular}{@{}p{20pt}p{298pt}@{}}
{\tt\%z}&The offset from GMT in ISO 8601 format.\\
\end{tabular}

\medskip

\begin{tabular}{@{}p{20pt}p{298pt}@{}}
{\tt\%\%}&A literal {\tt\%} character.\\
\end{tabular}

\medskip

All other (ordinary) characters in the format string must have a
matching character in the input string to be converted. Exceptions are
spaces in the input string which can match zero or more space
characters in the format string.

If some date and/or time component(s) are missing in the format and,
therefore, in the input string, the function {\tt str2time} uses their
default values corresponding to 00:00:00 on January 1, 1970, that is,
the default value of the year is 1970, the default value of the month
is January, etc.

The function {\tt str2time} is applicable to all calendar times in the
range 00:00:00 on January 1, 0001 to 23:59:59 on December 31, 4000 of
the Gregorian calendar.

\subsection{Converting calendar time to character string}
\label{time2str}

The function {\tt time2str(}{\it t}{\tt,} {\it f}{\tt)} converts the
calendar time specified by its first argument {\it t}, which must be a
numeric expression, to a character string (symbolic value). The
conversion is controlled by the specified format string {\it f} (the
second argument), which must be a symbolic expression.

The calendar time passed to {\tt time2str} has the same meaning as
values returned by the function {\tt gmtime} (see Subsection
\ref{gmtime}, page \pageref{gmtime}). Note that {\tt time2str} does
{\it not} correct the specified calendar time for the local timezone,
i.e. the calendar time 0 always corresponds to 00:00:00 on January 1,
1970.

For example, the model statements:

\medskip

\verb|   param s, symbolic, := time2str(gmtime(), "%FT%TZ");|

\verb|   display s;|

\medskip

\noindent may produce the following printout:

\medskip

\verb|   s = '2008-12-04T00:23:45Z'|

\medskip

\noindent which is a timestamp in the ISO format.

The format string passed to the function {\tt time2str} consists of
conversion specifiers and ordinary characters. Each conversion
specifier begins with a percent ({\tt\%}) character followed by a
letter.

The following conversion specifiers may be used in the format string:

\medskip

\begin{tabular}{@{}p{20pt}p{298pt}@{}}
{\tt\%a}&The abbreviated (2-character) weekday name.\\
\end{tabular}

\medskip

\begin{tabular}{@{}p{20pt}p{298pt}@{}}
{\tt\%A}&The full weekday name.\\
\end{tabular}

\medskip

\begin{tabular}{@{}p{20pt}p{298pt}@{}}
{\tt\%b}&The abbreviated (3-character) month name.\\
\end{tabular}

\medskip

\begin{tabular}{@{}p{20pt}p{298pt}@{}}
{\tt\%B}&The full month name.\\
\end{tabular}

\medskip

\begin{tabular}{@{}p{20pt}p{298pt}@{}}
{\tt\%C}&The century of the year, that is the greatest integer not
greater than the year divided by 100.\\
\end{tabular}

\medskip

\begin{tabular}{@{}p{20pt}p{298pt}@{}}
{\tt\%d}&The day of the month as a decimal number (range 01 to 31).\\
\end{tabular}

\medskip

\begin{tabular}{@{}p{20pt}p{298pt}@{}}
{\tt\%D}&The date using the format \verb|%m/%d/%y|.\\
\end{tabular}

\medskip

\begin{tabular}{@{}p{20pt}p{298pt}@{}}
{\tt\%e}&The day of the month like with \verb|%d|, but padded with
blank rather than zero.\\
\end{tabular}

\medskip

\begin{tabular}{@{}p{20pt}p{298pt}@{}}
{\tt\%F}&The date using the format \verb|%Y-%m-%d|.\\
\end{tabular}

\medskip

\begin{tabular}{@{}p{20pt}p{298pt}@{}}
{\tt\%g}&The year corresponding to the ISO week number, but without the
century (range 00 to 99). This has the same format and value as
\verb|%y|, except that if the ISO week number (see \verb|%V|) belongs
to the previous or next year, that year is used instead.\\
\end{tabular}

\medskip

\begin{tabular}{@{}p{20pt}p{298pt}@{}}
{\tt\%G}&The year corresponding to the ISO week number. This has the
same format and value as \verb|%Y|, except that if the ISO week number
(see \verb|%V|) belongs to the previous or next year, that year is used
instead.
\end{tabular}

\medskip

\begin{tabular}{@{}p{20pt}p{298pt}@{}}
{\tt\%h}&The same as \verb|%b|.\\
\end{tabular}

\medskip

\begin{tabular}{@{}p{20pt}p{298pt}@{}}
{\tt\%H}&The hour as a decimal number, using a 24-hour clock (range 00
to 23).\\
\end{tabular}

\medskip

\begin{tabular}{@{}p{20pt}p{298pt}@{}}
{\tt\%I}&The hour as a decimal number, using a 12-hour clock (range 01
to 12).\\
\end{tabular}

\medskip

\begin{tabular}{@{}p{20pt}p{298pt}@{}}
{\tt\%j}&The day of the year as a decimal number (range 001 to 366).\\
\end{tabular}

\medskip

\begin{tabular}{@{}p{20pt}p{298pt}@{}}
{\tt\%k}&The hour as a decimal number, using a 24-hour clock like
\verb|%H|, but padded with blank rather than zero.\\
\end{tabular}

\medskip

\begin{tabular}{@{}p{20pt}p{298pt}@{}}
{\tt\%l}&The hour as a decimal number, using a 12-hour clock like
\verb|%I|, but padded with blank rather than zero.
\end{tabular}

\medskip

\begin{tabular}{@{}p{20pt}p{298pt}@{}}
{\tt\%m}&The month as a decimal number (range 01 to 12).\\
\end{tabular}

\medskip

\begin{tabular}{@{}p{20pt}p{298pt}@{}}
{\tt\%M}&The minute as a decimal number (range 00 to 59).\\
\end{tabular}

\medskip

\begin{tabular}{@{}p{20pt}p{298pt}@{}}
{\tt\%p}&Either {\tt AM} or {\tt PM}, according to the given time value.
Midnight is treated as {\tt AM} and noon as {\tt PM}.\\
\end{tabular}

\medskip

\begin{tabular}{@{}p{20pt}p{298pt}@{}}
{\tt\%P}&Either {\tt am} or {\tt pm}, according to the given time value.
Midnight is treated as {\tt am} and noon as {\tt pm}.\\
\end{tabular}

\medskip

\begin{tabular}{@{}p{20pt}p{298pt}@{}}
{\tt\%R}&The hour and minute in decimal numbers using the format
\verb|%H:%M|.\\
\end{tabular}

\medskip

\begin{tabular}{@{}p{20pt}p{298pt}@{}}
{\tt\%S}&The second as a decimal number (range 00 to 59).\\
\end{tabular}

\medskip

\begin{tabular}{@{}p{20pt}p{298pt}@{}}
{\tt\%T}&The time of day in decimal numbers using the format
\verb|%H:%M:%S|.\\
\end{tabular}

\medskip

\begin{tabular}{@{}p{20pt}p{298pt}@{}}
{\tt\%u}&The day of the week as a decimal number (range 1 to 7), Monday
being 1.\\
\end{tabular}

\medskip

\begin{tabular}{@{}p{20pt}p{298pt}@{}}
{\tt\%U}&The week number of the current year as a decimal number (range
00 to 53), starting with the first Sunday as the first day of the first
week. Days preceding the first Sunday in the year are considered to be
in week 00.
\end{tabular}

\medskip

\begin{tabular}{@{}p{20pt}p{298pt}@{}}
{\tt\%V}&The ISO week number as a decimal number (range 01 to 53). ISO
weeks start with Monday and end with Sunday. Week 01 of a year is the
first week which has the majority of its days in that year; this is
equivalent to the week containing January 4. Week 01 of a year can
contain days from the previous year. The week before week 01 of a year
is the last week (52 or 53) of the previous year even if it contains
days from the new year. In other word, if 1 January is Monday, Tuesday,
Wednesday or Thursday, it is in week 01; if 1 January is Friday,
Saturday or Sunday, it is in week 52 or 53 of the previous year.\\
\end{tabular}

\medskip

\begin{tabular}{@{}p{20pt}p{298pt}@{}}
{\tt\%w}&The day of the week as a decimal number (range 0 to 6), Sunday
being 0.\\
\end{tabular}

\medskip

\begin{tabular}{@{}p{20pt}p{298pt}@{}}
{\tt\%W}&The week number of the current year as a decimal number (range
00 to 53), starting with the first Monday as the first day of the first
week. Days preceding the first Monday in the year are considered to be
in week 00.\\
\end{tabular}

\medskip

\begin{tabular}{@{}p{20pt}p{298pt}@{}}
{\tt\%y}&The year without a century as a decimal number (range 00 to
99), that is the year modulo 100.\\
\end{tabular}

\medskip

\begin{tabular}{@{}p{20pt}p{298pt}@{}}
{\tt\%Y}&The year as a decimal number, using the Gregorian calendar.\\
\end{tabular}

\medskip

\begin{tabular}{@{}p{20pt}p{298pt}@{}}
{\tt\%\%}&A literal \verb|%| character.\\
\end{tabular}

\medskip

All other (ordinary) characters in the format string are simply copied
to the resultant string.

The first argument (calendar time) passed to the function {\tt time2str}
must be in the range from $-62135596800$ to $+64092211199$ that
corresponds to the period from 00:00:00 on January 1, 0001 to 23:59:59
on December 31, 4000 of the Gregorian calendar.

%%%%%%%%%%%%%%%%%%%%%%%%%%%%%%%%%%%%%%%%%%%%%%%%%%%%%%%%%%%%%%%%%%%%%%%%

\newpage

\section{Table drivers}
\label{drivers}

\noindent\hfil
by Andrew Makhorin \verb|<mao@gnu.org>|

\noindent\hfil
and Heinrich Schuchardt \verb|<heinrich.schuchardt@gmx.de>|

\bigskip\bigskip

The {\it table driver} is a program module which provides transmitting
data between MathProg model objects and data tables.

Currently the GLPK package has four table drivers:

\setlength{\leftmargini}{2.5em}

\begin{itemize}
\item built-in CSV table driver;
\item built-in xBASE table driver;
\item ODBC table driver;
\item MySQL table driver.
\end{itemize}

\subsection{CSV table driver}

The CSV table driver assumes that the data table is represented in the
form of a plain text file in the CSV (comma-separated values) file
format as described below.

To choose the CSV table driver its name in the table statement should
be specified as \verb|"CSV"|, and the only argument should specify the
name of a plain text file containing the table. For example:

\medskip

\verb|   table data IN "CSV" "data.csv": ... ;|

\medskip

The filename suffix may be arbitrary, however, it is recommended to use
the suffix `\verb|.csv|'.

On reading input tables the CSV table driver provides an implicit field
named \verb|RECNO|, which contains the current record number. This
field can be specified in the input table statement as if there were
the actual field having the name \verb|RECNO| in the CSV file. For
example:

\medskip

\verb|   table list IN "CSV" "list.csv": num <- [RECNO], ... ;|

\subsubsection*{CSV format\footnote{This material is based on the RFC
document 4180.}}

The CSV (comma-separated values) format is a plain text file format
defined as follows.

1. Each record is located on a separate line, delimited by a line
break. For example:

\medskip

\verb|   aaa,bbb,ccc\n|

\verb|   xxx,yyy,zzz\n|

\medskip

\noindent
where \verb|\n| means the control character \verb|LF| ({\tt 0x0A}).

\newpage

2. The last record in the file may or may not have an ending line
break. For example:

\medskip

\verb|   aaa,bbb,ccc\n|

\verb|   xxx,yyy,zzz|

\medskip

3. There should be a header line appearing as the first line of the
file in the same format as normal record lines. This header should
contain names corresponding to the fields in the file. The number of
field names in the header line should be the same as the number of
fields in the records of the file. For example:

\medskip

\verb|   name1,name2,name3\n|

\verb|   aaa,bbb,ccc\n|

\verb|   xxx,yyy,zzz\n|

\medskip

4. Within the header and each record there may be one or more fields
separated by commas. Each line should contain the same number of fields
throughout the file. Spaces are considered as part of a field and
therefore not ignored. The last field in the record should not be
followed by a comma. For example:

\medskip

\verb|   aaa,bbb,ccc\n|

\medskip

5. Fields may or may not be enclosed in double quotes. For example:

\medskip

\verb|   "aaa","bbb","ccc"\n|

\verb|   zzz,yyy,xxx\n|

\medskip

6. If a field is enclosed in double quotes, each double quote which is
part of the field should be coded twice. For example:

\medskip

\verb|   "aaa","b""bb","ccc"\n|

\medskip

\noindent{\bf Example}

\begin{verbatim}
FROM,TO,DISTANCE,COST
Seattle,New-York,2.5,0.12
Seattle,Chicago,1.7,0.08
Seattle,Topeka,1.8,0.09
San-Diego,New-York,2.5,0.15
San-Diego,Chicago,1.8,0.10
San-Diego,Topeka,1.4,0.07
\end{verbatim}

\subsection{xBASE table driver}

The xBASE table driver assumes that the data table is stored in the
.dbf file format.

To choose the xBASE table driver its name in the table statement should
be specified as \verb|"xBASE"|, and the first argument should specify
the name of a .dbf file containing the table. For the output table there
should be the second argument defining the table format in the form
\verb|"FF...F"|, where \verb|F| is either {\tt C({\it n})},
which specifies a character field of length $n$, or
{\tt N({\it n}{\rm [},{\it p}{\rm ]})}, which specifies a numeric field
of length $n$ and precision $p$ (by default $p$ is 0).

The following is a simple example which illustrates creating and
reading a .dbf file:

\begin{verbatim}
table tab1{i in 1..10} OUT "xBASE" "foo.dbf"
   "N(5)N(10,4)C(1)C(10)": 2*i+1 ~ B, Uniform(-20,+20) ~ A,
   "?" ~ FOO, "[" & i & "]" ~ C;
set S, dimen 4;
table tab2 IN "xBASE" "foo.dbf": S <- [B, C, RECNO, A];
display S;
end;
\end{verbatim}

\subsection{ODBC table driver}

The ODBC table driver allows connecting to SQL databases using an
implementation of the ODBC interface based on the Call Level Interface
(CLI).\footnote{The corresponding software standard is defined in
ISO/IEC 9075-3:2003.}

\paragraph{Debian GNU/Linux.}
Under Debian GNU/Linux the ODBC table driver uses the iODBC
package,\footnote{See {\tt<http://www.iodbc.org/>}.} which should be
installed before building the GLPK package. The installation can be
effected with the following command:

\begin{verbatim}
sudo apt-get install libiodbc2-dev
\end{verbatim}

Note that on configuring the GLPK package to enable using the iODBC
library the option `\verb|--enable-odbc|' should be passed to the
configure script.

The individual databases must be entered for systemwide usage in
\linebreak \verb|/etc/odbc.ini| and \verb|/etc/odbcinst.ini|. Database
connections to be used by a single user are specified by files in the
home directory (\verb|.odbc.ini| and \verb|.odbcinst.ini|).

\paragraph{Microsoft Windows.}
Under Microsoft Windows the ODBC table driver uses the Microsoft ODBC
library. To enable this feature the symbol:

\begin{verbatim}
#define ODBC_DLNAME "odbc32.dll"
\end{verbatim}

\noindent
should be defined in the GLPK configuration file `\verb|config.h|'.

Data sources can be created via the Administrative Tools from the
Control Panel.

\bigskip

To choose the ODBC table driver its name in the table statement should
be specified as \verb|'ODBC'| or \verb|'iODBC'|.

The argument list is specified as follows.

The first argument is the connection string passed to the ODBC library,
for example:

\verb|'DSN=glpk;UID=user;PWD=password'|, or

\verb|'DRIVER=MySQL;DATABASE=glpkdb;UID=user;PWD=password'|.

Different parts of the string are separated by semicolons. Each part
consists of a pair {\it fieldname} and {\it value} separated by the
equal sign. Allowable fieldnames depend on the ODBC library. Typically
the following fieldnames are allowed:

\verb|DATABASE | database;

\verb|DRIVER   | ODBC driver;

\verb|DSN      | name of a data source;

\verb|FILEDSN  | name of a file data source;

\verb|PWD      | user password;

\verb|SERVER   | database;

\verb|UID      | user name.

The second argument and all following are considered to be SQL
statements

SQL statements may be spread over multiple arguments.  If the last
character of an argument is a semicolon this indicates the end of
a SQL statement.

The arguments of a SQL statement are concatenated separated by space.
The eventual trailing semicolon will be removed.

All but the last SQL statement will be executed directly.

For IN-table the last SQL statement can be a SELECT command starting
with the capitalized letters \verb|'SELECT '|. If the string does not
start with \verb|'SELECT '| it is considered to be a table name and a
SELECT statement is automatically generated.

For OUT-table the last SQL statement can contain one or multiple
question marks. If it contains a question mark it is considered a
template for the write routine. Otherwise the string is considered a
table name and an INSERT template is automatically generated.

The writing routine uses the template with the question marks and
replaces the first question mark by the first output parameter, the
second question mark by the second output parameter and so forth. Then
the SQL command is issued.

The following is an example of the output table statement:

\begin{small}
\begin{verbatim}
table ta { l in LOCATIONS } OUT
   'ODBC'
   'DSN=glpkdb;UID=glpkuser;PWD=glpkpassword'
   'DROP TABLE IF EXISTS result;'
   'CREATE TABLE result ( ID INT, LOC VARCHAR(255), QUAN DOUBLE );'
   'INSERT INTO result 'VALUES ( 4, ?, ? )' :
   l ~ LOC, quantity[l] ~ QUAN;
\end{verbatim}
\end{small}

\noindent
Alternatively it could be written as follows:

\begin{small}
\begin{verbatim}
table ta { l in LOCATIONS } OUT
   'ODBC'
   'DSN=glpkdb;UID=glpkuser;PWD=glpkpassword'
   'DROP TABLE IF EXISTS result;'
   'CREATE TABLE result ( ID INT, LOC VARCHAR(255), QUAN DOUBLE );'
   'result' :
   l ~ LOC, quantity[l] ~ QUAN, 4 ~ ID;
\end{verbatim}
\end{small}

Using templates with `\verb|?|' supports not only INSERT, but also
UPDATE, DELETE, etc. For example:

\begin{small}
\begin{verbatim}
table ta { l in LOCATIONS } OUT
   'ODBC'
   'DSN=glpkdb;UID=glpkuser;PWD=glpkpassword'
   'UPDATE result SET DATE = ' & date & ' WHERE ID = 4;'
   'UPDATE result SET QUAN = ? WHERE LOC = ? AND ID = 4' :
   quantity[l], l;
\end{verbatim}
\end{small}

\subsection{MySQL table driver}

The MySQL table driver allows connecting to MySQL databases.

\paragraph{Debian GNU/Linux.}
Under Debian GNU/Linux the MySQL table\linebreak driver uses the MySQL
package,\footnote{For download development files see
{\tt<http://dev.mysql.com/downloads/mysql/>}.} which should be installed
before building the GLPK package. The installation can be effected with
the following command:

\begin{verbatim}
sudo apt-get install libmysqlclient15-dev
\end{verbatim}

Note that on configuring the GLPK package to enable using the MySQL
library the option `\verb|--enable-mysql|' should be passed to the
configure script.

\paragraph{Microsoft Windows.}
Under Microsoft Windows the MySQL table driver also uses the MySQL
library. To enable this feature the symbol:

\begin{verbatim}
#define MYSQL_DLNAME "libmysql.dll"
\end{verbatim}

\noindent
should be defined in the GLPK configuration file `\verb|config.h|'.

\bigskip

To choose the MySQL table driver its name in the table statement should
be specified as \verb|'MySQL'|.

The argument list is specified as follows.

The first argument specifies how to connect the data base in the DSN
style, for example:

\verb|'Database=glpk;UID=glpk;PWD=gnu'|.

Different parts of the string are separated by semicolons. Each part
consists of a pair {\it fieldname} and {\it value} separated by the
equal sign. The following fieldnames are allowed:

\verb|Server   | server running the database (defaulting to localhost);

\verb|Database | name of the database;

\verb|UID      | user name;

\verb|PWD      | user password;

\verb|Port     | port used by the server (defaulting to 3306).

The second argument and all following are considered to be SQL
statements

SQL statements may be spread over multiple arguments.  If the last
character of an argument is a semicolon this indicates the end of
a SQL statement.

The arguments of a SQL statement are concatenated separated by space.
The eventual trailing semicolon will be removed.

All but the last SQL statement will be executed directly.

For IN-table the last SQL statement can be a SELECT command starting
with the capitalized letters \verb|'SELECT '|. If the string does not
start with \verb|'SELECT '| it is considered to be a table name and a
SELECT statement is automatically generated.

For OUT-table the last SQL statement can contain one or multiple
question marks. If it contains a question mark it is considered a
template for the write routine. Otherwise the string is considered a
table name and an INSERT template is automatically generated.

The writing routine uses the template with the question marks and
replaces the first question mark by the first output parameter, the
second question mark by the second output parameter and so forth. Then
the SQL command is issued.

The following is an example of the output table statement:

\begin{small}
\begin{verbatim}
table ta { l in LOCATIONS } OUT
   'MySQL'
   'Database=glpkdb;UID=glpkuser;PWD=glpkpassword'
   'DROP TABLE IF EXISTS result;'
   'CREATE TABLE result ( ID INT, LOC VARCHAR(255), QUAN DOUBLE );'
   'INSERT INTO result VALUES ( 4, ?, ? )' :
   l ~ LOC, quantity[l] ~ QUAN;
\end{verbatim}
\end{small}

\noindent
Alternatively it could be written as follows:

\begin{small}
\begin{verbatim}
table ta { l in LOCATIONS } OUT
   'MySQL'
   'Database=glpkdb;UID=glpkuser;PWD=glpkpassword'
   'DROP TABLE IF EXISTS result;'
   'CREATE TABLE result ( ID INT, LOC VARCHAR(255), QUAN DOUBLE );'
   'result' :
   l ~ LOC, quantity[l] ~ QUAN, 4 ~ ID;
\end{verbatim}
\end{small}

Using templates with `\verb|?|' supports not only INSERT, but also
UPDATE, DELETE, etc. For example:

\begin{small}
\begin{verbatim}
table ta { l in LOCATIONS } OUT
   'MySQL'
   'Database=glpkdb;UID=glpkuser;PWD=glpkpassword'
   'UPDATE result SET DATE = ' & date & ' WHERE ID = 4;'
   'UPDATE result SET QUAN = ? WHERE LOC = ? AND ID = 4' :
   quantity[l], l;
\end{verbatim}
\end{small}

%%%%%%%%%%%%%%%%%%%%%%%%%%%%%%%%%%%%%%%%%%%%%%%%%%%%%%%%%%%%%%%%%%%%%%%%

\newpage

\section{Solving models with glpsol}

The GLPK package\footnote{{\tt http://www.gnu.org/software/glpk/}}
includes the program {\tt glpsol}, which is a stand-alone LP/MIP solver.
This program can be launched from the command line or from the shell to
solve models written in the GNU MathProg modeling language.

In order to tell the solver that the input file contains a model
description, you need to specify the option \verb|--model| in the
command line. For example:

\medskip

\verb|   glpsol --model foo.mod|

\medskip

Sometimes it is necessary to use the data section placed in a separate
file, in which case you may use the following command:

\medskip

\verb|   glpsol --model foo.mod --data foo.dat|

\medskip

\noindent Note that if the model file also contains the data section,
that section is ignored.

If the model description contains some display and/or printf statements,
by default the output is sent to the terminal. In order to redirect the
output to a file you may use the following command:

\medskip

\verb|   glpsol --model foo.mod --display foo.out|

\medskip

If you need to look at the problem, which has been generated by the
model translator, you may use the option \verb|--wlp| as follows:

\medskip

\verb|   glpsol --model foo.mod --wlp foo.lp|

\medskip

\noindent in which case the problem data is written to file
\verb|foo.lp| in CPLEX LP format suitable for visual analysis.

Sometimes it is needed merely to check the model description not
solving the generated problem instance. In this case you may specify
the option \verb|--check|, for example:

\medskip

\verb|   glpsol --check --model foo.mod --wlp foo.lp|

\medskip

In order to write a numeric solution obtained by the solver you may use
the following command:

\medskip

\verb|   glpsol --model foo.mod --output foo.sol|

\medskip

\noindent in which case the solution is written to file \verb|foo.sol|
in a plain text format.

The complete list of the \verb|glpsol| options can be found in the
reference manual included in the GLPK distribution.

%%%%%%%%%%%%%%%%%%%%%%%%%%%%%%%%%%%%%%%%%%%%%%%%%%%%%%%%%%%%%%%%%%%%%%%%

\newpage

\section{Example model description}

\subsection{Model description written in MathProg}

Below here is a complete example of the model description written in
the GNU MathProg modeling language.

\begin{small}
\begin{verbatim}
# A TRANSPORTATION PROBLEM
#
# This problem finds a least cost shipping schedule that meets
# requirements at markets and supplies at factories.
#
#  References:
#              Dantzig G B, "Linear Programming and Extensions."
#              Princeton University Press, Princeton, New Jersey, 1963,
#              Chapter 3-3.

set I;
/* canning plants */

set J;
/* markets */

param a{i in I};
/* capacity of plant i in cases */

param b{j in J};
/* demand at market j in cases */

param d{i in I, j in J};
/* distance in thousands of miles */

param f;
/* freight in dollars per case per thousand miles */

param c{i in I, j in J} := f * d[i,j] / 1000;
/* transport cost in thousands of dollars per case */

var x{i in I, j in J} >= 0;
/* shipment quantities in cases */

minimize cost: sum{i in I, j in J} c[i,j] * x[i,j];
/* total transportation costs in thousands of dollars */

s.t. supply{i in I}: sum{j in J} x[i,j] <= a[i];
/* observe supply limit at plant i */

s.t. demand{j in J}: sum{i in I} x[i,j] >= b[j];
/* satisfy demand at market j */

data;

set I := Seattle San-Diego;

set J := New-York Chicago Topeka;

param a := Seattle     350
           San-Diego   600;

param b := New-York    325
           Chicago     300
           Topeka      275;

param d :              New-York   Chicago   Topeka :=
           Seattle     2.5        1.7       1.8
           San-Diego   2.5        1.8       1.4  ;

param f := 90;

end;
\end{verbatim}
\end{small}

\subsection{Generated LP problem instance}

Below here is the result of the translation of the example model
produced by the solver \verb|glpsol| and written in CPLEX LP format
with the option \verb|--wlp|.

\begin{small}
\begin{verbatim}
\* Problem: transp *\

Minimize
 cost: + 0.225 x(Seattle,New~York) + 0.153 x(Seattle,Chicago)
 + 0.162 x(Seattle,Topeka) + 0.225 x(San~Diego,New~York)
 + 0.162 x(San~Diego,Chicago) + 0.126 x(San~Diego,Topeka)

Subject To
 supply(Seattle): + x(Seattle,New~York) + x(Seattle,Chicago)
 + x(Seattle,Topeka) <= 350
 supply(San~Diego): + x(San~Diego,New~York) + x(San~Diego,Chicago)
 + x(San~Diego,Topeka) <= 600
 demand(New~York): + x(Seattle,New~York) + x(San~Diego,New~York) >= 325
 demand(Chicago): + x(Seattle,Chicago) + x(San~Diego,Chicago) >= 300
 demand(Topeka): + x(Seattle,Topeka) + x(San~Diego,Topeka) >= 275

End
\end{verbatim}
\end{small}

\subsection{Optimal LP solution}

Below here is the optimal solution of the generated LP problem instance
found by the solver \verb|glpsol| and written in plain text format
with the option \verb|--output|.

\newpage

\begin{small}
\begin{verbatim}
Problem:    transp
Rows:       6
Columns:    6
Non-zeros:  18
Status:     OPTIMAL
Objective:  cost = 153.675 (MINimum)

No.   Row name   St   Activity    Lower bound  Upper bound   Marginal
--- ------------ -- ------------ ------------ ------------ ------------
  1 cost         B       153.675
  2 supply[Seattle]
                 B           300                       350
  3 supply[San-Diego]
                 NU          600                       600        < eps
  4 demand[New-York]
                 NL          325          325                     0.225
  5 demand[Chicago]
                 NL          300          300                     0.153
  6 demand[Topeka]
                 NL          275          275                     0.126

No. Column name  St   Activity    Lower bound  Upper bound   Marginal
--- ------------ -- ------------ ------------ ------------ ------------
  1 x[Seattle,New-York]
                 B             0            0
  2 x[Seattle,Chicago]
                 B           300            0
  3 x[Seattle,Topeka]
                 NL            0            0                     0.036
  4 x[San-Diego,New-York]
                 B           325            0
  5 x[San-Diego,Chicago]
                 NL            0            0                     0.009
  6 x[San-Diego,Topeka]
                 B           275            0

End of output
\end{verbatim}
\end{small}

%%%%%%%%%%%%%%%%%%%%%%%%%%%%%%%%%%%%%%%%%%%%%%%%%%%%%%%%%%%%%%%%%%%%%%%%

\newpage

\setcounter{secnumdepth}{-1}

\section{Acknowledgment}

The authors would like to thank the following people, who kindly read,
commented, and corrected the draft of this document:

\medskip

\noindent Juan Carlos Borras \verb|<borras@cs.helsinki.fi>|

\medskip

\noindent Harley Mackenzie \verb|<hjm@bigpond.com>|

\medskip

\noindent Robbie Morrison \verb|<robbie@actrix.co.nz>|

\end{document}
