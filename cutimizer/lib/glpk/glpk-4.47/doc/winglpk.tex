%* winglpk.tex *%

%***********************************************************************
%  This code is part of GLPK for Windows.
%
%  Copyright (C) 2009 Heinrich Schuchardt, <xypron.glpk@gmx.de>
%
%  GLPK for Windows is free software: you can redistribute it and/or 
%  modify it under the terms of the GNU General Public License as 
%  published by the Free Software Foundation, either version 3 of 
%  the License, or (at your option) any later version.
%
%  GLPK for Windows is distributed in the hope that it will be useful, 
%  but WITHOUT ANY WARRANTY; without even the implied warranty of 
%  MERCHANTABILITY or FITNESS FOR A PARTICULAR PURPOSE. See the GNU 
%  General Public License for more details.
%
%  You should have received a copy of the GNU General Public License
%  along with GLPK-Java. If not, see <http://www.gnu.org/licenses/>.
%***********************************************************************

%\documentclass[dvipdfm,11pt]{report}
%\usepackage[dvipdfm,linktocpage,colorlinks,linkcolor=blue,urlcolor=blue]{hyperref}
\documentclass[a4paper,11pt]{report}
\usepackage[all]{xy}
\usepackage{hyperref}
\usepackage{parskip}

\renewcommand\contentsname{\sf\bfseries Contents}
\renewcommand\chaptername{\sf\bfseries Chapter}
\renewcommand\appendixname{\sf\bfseries Appendix}

\setlength{\parindent}{0pt}
\setlength{\parskip}{10pt} 

\begin{document}

\thispagestyle{empty}

\begin{center}

\vspace*{1in}

\begin{huge}
\sf\bfseries GNU Linear Programming Kit\linebreak
for Windows
\end{huge}

\vspace{0.5in}

\begin{LARGE}
\sf Reference Manual
\end{LARGE}

\vspace{0.5in}

\begin{LARGE}
\sf Version 4.40
\end{LARGE}

\vspace{0.5in}
\begin{Large}
\sf October 2009
\end{Large}
\end{center}

\newpage

\vspace*{1in}

\vfill

\medskip \noindent
Copyright \copyright{} 2009 Heinrich Schuchardt, xypron.glpk@gmx.de

\medskip \noindent
Permission is granted to make and distribute verbatim copies of this
manual provided the copyright notice and this permission notice are
preserved on all copies.

\medskip \noindent
Permission is granted to copy and distribute modified versions of this
manual under the conditions for verbatim copying, provided also that the
entire resulting derived work is distributed under the terms of
a permission notice identical to this one.

\medskip \noindent
Permission is granted to copy and distribute translations of this manual
into another language, under the above conditions for modified versions.

\medskip \noindent
Windows is a registered trademark of Microsoft Corporation. Java is a 
trademark when it identifies a software product of Sun Microsystems, Inc.

\tableofcontents

\chapter{Introduction}
The GNU Linear Programming Kit (GLPK) package supplies a solver for large scale linear programming (LP) and mixed integer programming (MIP). The GLPK project is hosted at \href{http://www.gnu.org/software/glpk}{http://www.gnu.org/software/glpk}.

It has two mailing lists: 
\begin{itemize}
\item\href{mailto:help-glpk@gnu.org}{help-glpk@gnu.org} and 
\item\href{mailto:bug-glpk@gnu.org}{bug-glpk@gnu.org}.
\end{itemize}
To subscribe to one of these lists, please, send an empty mail with a Subject: header line of just "subscribe" to the list.

GLPK provides a library written in C and a standalone solver.

The source code provided at \href{ftp://gnu.ftp.org/gnu/glpk/}{ftp://gnu.ftp.org/gnu/glpk/} contains the documentation of the library in  file doc/glpk.pdf.

Project GLPK for Windows delivers precompiled executables for Windows. It is hosted at \href{http://winglpk.sourceforge.net/}{http://winglpk.sourceforge.net/}.

\chapter{Installation}
The GLPK for Windows package can be downloaded from \linebreak\href{http://winglpk.sourceforge.net/}{http://winglpk.sourceforge.net/}.

It is distributed as compressed archive. Please decompress it to your preferred installation path (e.g. c:\textbackslash program files\textbackslash glpk).

The executables and dynamic link libraries for 32 bit Windows can be found in directory w32, those for 64 bit Windows can be found in directory w64.

The libraries have to be in the search path for binaries when executing the standalone solver glpsol.exe. Therefore it is suggested to copy the DLLs to \%SystemRoot\%\textbackslash system32 (e.g. c:\textbackslash windows\textbackslash system32).

The documentation can be found in directory doc. If you want to use the GNU Math Programming Language (GMPL), please read gmpl.pdf and tables.pdf. If you want to use the library in your own coding, please, refer to glpk.pdf.

\chapter{Building}
As GLPK for Windows comes with precompiled binaries it is normally not necessary to build the software.

The script used to build the package is provided as Build\_WinGLPK.bat with the distribution.

The following software is needed
\begin{itemize}
\item Windows Vista 64bit
\item Java SE Development Kit (JDK) available at \linebreak\href{http://java.sun.com/javase/downloads/index.jsp}{http://java.sun.com/javase/downloads/index.jsp}.
\item Microsoft Visual C++ 2008 Express Edition available at \linebreak\href{http://www.microsoft.com/downloads/en/default.aspx}{http://www.microsoft.com/downloads/en/default.aspx}
\item Windows Software Development Kit (SDK) for Windows Server 2008 available at \href{http://www.microsoft.com/downloads/en/default.aspx}{http://www.microsoft.com/downloads/en/default.aspx}
\item GnuWin32 available at \href{http://gnuwin32.sourceforge.net/}{http://gnuwin32.sourceforge.net/}
\item SWIG (Simplified Wrapper and Interface Generator) available at \linebreak\href{http://www.swig.org}{http://www.swig.org}
\end{itemize}
When building, please, create a new directory and copy only Build\_WinGLPK.bat to the directory. In the command shell execute the script. The distribution files are created in subdirectory release.
\end{document}
